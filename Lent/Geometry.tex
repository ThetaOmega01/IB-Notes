\documentclass[a4paper]{article}
\renewcommand{\epsilon}{\varepsilon}
\newcommand{\triposcourse}{Geometry}
\input{../header.tex}
\counterwithin{equation}{section}
\graphicspath{ {./images/} }
\pgfplotsset{compat=1.17}
\begin{document}
\maketitle
\tableofcontents
\clearpage

\section{Surfaces}
Let's start with a definition.
\begin{definition}
    A \textbf{topological surface} is a topological space $ \Sigma $ such that 
    \begin{enumerate}[(a)]
        \item $ \forall \rho\in \Sigma $, there is an open neighbourhood $\rho\in U \subset \Sigma$ such that $U$ is homeomorphic to $\mathbb{R}^2$, or a disc $D^2 \subset \mathbb{R}^{2}$, with its usual Euclidean topology. 
        \item $\Sigma$ is Hausdorff and second countable. 
    \end{enumerate}
\end{definition}
\begin{remark}
    \begin{enumerate}[(1)]
        \item Write $ \cong $ for homeomorphic, $ \mathbb{R}^{2} \cong D^2(0,1) = \{x\in \mathbb{R}^{2}: \left\| x \right\|<1\} $

        \item A space $X$ is Hausdorff if for $p\neq q$ in $X$, there exists disjoint open sets $U\ni p, V\ni q$ in $X$. 
        A space is second countable if it has a countable base, i.e. there exists $ \{U_i\}_{i\in \mathbb{N}} $ open sets such that every open set is a union of some $U_i$. (a) is the point and (b) is for technical honesty.
             
        \item If $X$ is Hausdorff or second countable, so are subspaces of $X$. Euclidean space has these properties. To see it is second countable, consider open balls $B(c,r)$ with $c\in \mathbb{Q}^n$ and $r\in \mathbb{Q}_+$. 
    \end{enumerate}
\end{remark}

\begin{example}
    \begin{enumerate}
        \item $ \mathbb{R}^{2}$ the plane. 
        \item Any open set in $ \mathbb{R}^{2}$, i.e. $ \mathbb{R}^{2}\setminus Z $ where $Z$ is closed, e.g. $ Z = \{0\} $ or $ Z = \{(0,0) \cup (0,\frac{1}{n}), n=1,2,\dots\} $. 
    \end{enumerate}
\end{example}

\begin{example}
    Let \( f \colon \mathbb R^2 \to \mathbb R \) be a continuous function.
	The graph of \( f \), denoted \( \Gamma_f \), is defined by
	\[
		\Gamma_f = \qty{(x,y,f(x,y)) \colon (x,y) \in \mathbb R^2}
	\]
	with the subspace topology when embedded in \( \mathbb R^3 \).

	Recall that a product topology \( X \times Y \) has the feature that \( f \colon Z \to X \times Y \) is continuous if and only if \( \pi_x \circ f \colon Z \to X \) and \( \pi_y \circ f \colon Z \to Y \) are continuous.

	Hence, any graph \( \Gamma \subseteq X \times Y \) is homeomorphic to \( X \) if \( f \) is continuous.
	Indeed, the projection \( \pi_x \) projects each point in the graph onto the domain.
	The function \( s \colon x \mapsto (x,f(x)) \) is continuous by the above.
	In particular, in our case, the graph \( \Gamma_f \) is homeomorphic to \( \mathbb R^2 \), which we know is a topological surface.
\end{example}
\begin{remark}
	As a topological surface, \( \Gamma_f \) is independent of the function \( f \).
	However, we will later introduce more ways to describe topological spaces that will ascribe new properties to \( \Gamma_f \) which do depend on \( f \).
\end{remark}
\begin{example}
	The sphere:
	\[
		S^2 = \qty{(x,y,z) \in \mathbb R^3 \colon x^2 + y^2 + z^2 = 1}
	\]
	is a topological surface, when using the subspace topology in \( \mathbb R^3 \).
	Consider the stereographic projection of \( S^2 \) onto \( \mathbb R^2 \) from the north pole \( (0,0,1) \).
	The projection satisfies \( \pi_+ \colon S^2 \setminus \qty{(0,0,1)} \) and
	\[
		(x,y,z) \mapsto \qty(\frac{x}{1-z}, \frac{y}{1-z})
	\]
	Certainly, \( \pi_+ \) is continuous, since we do not consider the point \( (0,0,1) \) to be in its domain.
	The inverse map is given by
	\[
		(u,v) \mapsto \qty(\frac{2u}{u^2+v^2+1}, \frac{2v}{u^2+v^2+1}, \frac{u^2+v^2-1}{u^2+v^2+1})
	\]
	This is also a continuous function.
	Hence \( \pi_+ \) is a homeomorphism.
	Similarly, we can construct the stereographic projection from the south pole, \( \pi_- \) defined by 
    \[
        (x,y,z) \mapsto \left( \frac{x}{1+z},\frac{y}{1+z} \right). 
    \]
	This is a homeomorphism.
	Hence, every point in \( S^2 \) lies either in the domain of \( \pi_+ \) or \( \pi_- \), and hence sits in an open set \( S^2 \setminus \qty{(0,0,1)} \) or \( S^2 \setminus \qty{(0,0,-1)} \) which is homeomorphic to \( \mathbb R^2 \).
\end{example}
\begin{note}
	\( S^2 \) is compact by the Heine-Borel theorem, that is, it is a closed bounded set in \( \mathbb R^3 \).
\end{note}

\begin{example}
	The real projective plane is a topological surface.
	The group \( \mathbb Z / 2 \) acts on \( S^2 \) by homeomorphisms via the \textbf{antipodal map} \( a \colon S^2 \to S^2 \), mapping \( x \mapsto -x \).
	There exists a homeomorphism \( \mathbb Z / 2 \) to the group \( \mathrm{Homeo}(S^2) \) of homeomorphisms of \( S^2 \), by mapping \( 1 + \mathbb Z \mapsto a \).

	We now define the real projective plane to be the quotient of \( S^2 \) given by identifying every point \( x \) with its image \( -x \) under \( a \).
	\[
		\mathbb R \mathbb P^2 = \faktor{S^2}{\mathbb Z/2} = \faktor{S^2}{\sim};\quad x \sim a(x)
	\]
	\begin{lemma}
		\( \mathbb R \mathbb P^2 \) bijects with the set of straight lines in \( \mathbb R^3 \) through the origin.
	\end{lemma}
	\begin{proof}
		Any line through the origin intersects \( S^2 \) exactly in a pair of antipodal points \( x, -x \).
		Similarly, pairs of antipodal points uniquely define a line through the origin.
	\end{proof}
	\begin{lemma}
		\( \mathbb R \mathbb P^2 \) is a topological surface with the quotient topology.
	\end{lemma}
	\begin{proof}
		We must check that \( \mathbb R \mathbb P^2 \) is Hausdorff since it is constructed by a quotient, not a subspace.
		If \( X \) is a space and \( q \colon X \to Y \) is a quotient map, then by definition \( V \subset Y \) is open if and only if \( q^{-1}(V) \subset X \) is open.
		If \( [p] \neq [q] \in \mathbb R \mathbb P^2 \), then \( \pm p, \pm q \in S^2 \) are distinct antipodal pairs.
		We can therefore construct distinct open discs around \( p, q \) in \( S^2 \), and their antipodal images.
		These uniquely define open neighbourhoods of \( [p], [q] \), which are disjoint.

		Similarly, we can check that \( \mathbb R \mathbb P^2 \) is second countable.
		We know that \( S^2 \) is second countable, so let \( \mathcal U \) be a countable base for the topology on \( S^2 \).
		Without loss of generality, we can assert that for all sets \( U \in \mathcal U \), we have \( -U \in \mathcal U \).
		Let \( \overline{\mathcal U} \) be the family of open sets in \( \mathbb R \mathbb P^2 \) of the form \( q(U) \cup q(-U) \) for \( U \in \mathcal U \), where \( q \) is the quotient map.
		Now, if \( V \subseteq \mathbb R \mathbb P^2 \) is open, then by definition \( q^{-1}(V) \) is open in \( S^2 \) hence \( q^{-1}(V) \) contains some \( U \in \mathcal U \) and hence contains \( U \cup (-U) \).
		Hence \( \overline{\mathcal U} \) is a countable base for the quotient topology on \( \mathbb R \mathbb P^2 \).

		Finally, let \( p \in S^2 \) and \( [p] \in \mathbb R \mathbb P^2 \) its image.
		Let \( \overline D \) be a small (contained in an open hemisphere) closed disc, which is a neighbourhood of \( p \in S^2 \).
		The quotient map restricted to \( \overline D \), written \( \eval{q}_{\overline D} \colon \overline D \to q(\overline D) \subset \mathbb R \mathbb P^2 \), is a continuous function from a compact space to a Hausdorff space.
		Further, \( q \) is injective on \( \overline D \) since the disc was contained entirely in a single hemisphere.
		The topological inverse function theorem states that a continuous bijection from a compact space to a Hausdorff space is a homeomorphism.
		So \( \eval{q}_{\overline D} \) is a homeomorphism from \( \overline D \) to \( q(\overline D) \).
		This then induces the homeomorphism \( \eval{q}_{D} \) from the open disc \( D = {\overline D}^\circ \) to \( q(D) \).
		So by construction, \( [p] \in q(D) \); it has an open neighbourhood in \( \mathbb R \mathbb P^2 \) which is homeomorphic to an open disc, concluding the proof.
	\end{proof}
\end{example}

\end{document}
\documentclass[a4paper]{article}
\renewcommand{\epsilon}{\varepsilon}
\newcommand{\triposcourse}{Geometry}
\usepackage{fancyhdr,titlesec,geometry}
\usepackage[dvipsnames]{xcolor}
\usepackage[many]{tcolorbox}
\usepackage{xifthen}
\usepackage{import}
\usepackage{parskip}
\usepackage{transparent}
\usepackage{mathtools,amssymb,amsfonts,amsthm,bm}   % Math Presets
\usepackage{array,tabularx,booktabs}                % Table Presets
\usepackage{graphicx,wrapfig,float,caption}         % Figure Presets
\usepackage{setspace,multicol}                      % Text Presets
\usepackage{tikz,physics,cancel,tkz-euclide,pgfplots,tikz-3dplot}                    % Physics Presets
\usepackage{amsmath}
\usepackage{mathrsfs}
\usepackage{enumerate}
\usepackage[shortlabels]{enumitem}
\usepackage{hyperref}
\usepackage{lipsum}
\usepackage{IEEEtrantools}
\usepackage{xcomment}
\usepackage{sectsty}
\usepackage{thmtools}
\usepackage{mdframed}
\usepackage{siunitx}
\usepackage{centernot}

\newcommand{\sectionbreak}{\clearpage}

\tdplotsetmaincoords{60}{120}

\usetikzlibrary{arrows.meta}
\usetikzlibrary{decorations.markings}
\usetikzlibrary{decorations.pathmorphing}
\usetikzlibrary{automata, positioning}
\usetikzlibrary{fadings}
\usetikzlibrary{intersections}
\usetikzlibrary{cd}
\usetikzlibrary{patterns}
\usetikzlibrary{shapes.arrows}
\usepgfplotslibrary{colormaps, external}
\pgfarrowsdeclarecombine{twolatex'}{twolatex'}{latex'}{latex'}{latex'}{latex'}
\tikzset{->/.style = {decoration={markings,
                                  mark=at position 1 with {\arrow[scale=1.6]{latex'}}},
                      postaction={decorate}}}
\tikzset{<-/.style = {decoration={markings,
                                  mark=at position 0 with {\arrowreversed[scale=1.6]{latex'}}},
                      postaction={decorate}}}
\tikzset{<->/.style = {decoration={markings,
                                   mark=at position 0 with {\arrowreversed[scale=1.6]{latex'}},
                                   mark=at position 1 with {\arrow[scale=1.6]{latex'}}},
                       postaction={decorate}}}
\tikzset{->-/.style = {decoration={markings,
                                   mark=at position #1 with {\arrow[scale=1.6]{latex'}}},
                       postaction={decorate}}}
\tikzset{-<-/.style = {decoration={markings,
                                   mark=at position #1 with {\arrowreversed[scale=1.6]{latex'}}},
                       postaction={decorate}}}
\tikzset{->>/.style = {decoration={markings,
                                  mark=at position 1 with {\arrow[scale=1.6]{twolatex'}}},
                      postaction={decorate}}}
\tikzset{<<-/.style = {decoration={markings,
                                  mark=at position 0 with {\arrowreversed[scale=1.6]{twolatex'}}},
                      postaction={decorate}}}
\tikzset{<<->>/.style = {decoration={markings,
                                   mark=at position 0 with {\arrowreversed[scale=1.6]{twolatex'}},
                                   mark=at position 1 with {\arrow[scale=1.6]{twolatex'}}},
                       postaction={decorate}}}
\tikzset{->>-/.style = {decoration={markings,
                                   mark=at position #1 with {\arrow[scale=1.6]{twolatex'}}},
                       postaction={decorate}}}
\tikzset{-<<-/.style = {decoration={markings,
                                   mark=at position #1 with {\arrowreversed[scale=1.6]{twolatex'}}},
                       postaction={decorate}}}

\tikzset{
set arrow inside/.code={\pgfqkeys{/tikz/arrow inside}{#1}},
set arrow inside={end/.initial=>, opt/.initial=},
/pgf/decoration/Mark/.style={
    mark/.expanded=at position #1 with
    {
        \noexpand\arrow[\pgfkeysvalueof{/tikz/arrow inside/opt}]{\pgfkeysvalueof{/tikz/arrow inside/end}}
    }
},
arrow inside/.style 2 args={
    set arrow inside={#1},
    postaction={
        decorate,decoration={
            markings,Mark/.list={#2}
        }
    }
},
}

\tikzstyle{circ}=[fill=black, draw=black, shape=circle]
\tikzset{
dot/.style = {circle, fill, minimum size=#1,
              inner sep=0pt, outer sep=0pt},
dot/.default = 5pt% size of the circle diameter 
}
\tikzset{mstate/.style={circle, draw, blue, text=black, minimum width=0.7cm}}
\tikzset{snake it/.style={-stealth,
decoration={snake, 
    amplitude = .4mm,
    segment length = 2mm,
    post length=0.9mm},decorate}}

\def\centerarc[#1](#2)(#3:#4:#5)% Syntax: [draw options](center)(initial angle:final angle:radius)
    { \draw[#1] ($(#2)+({#5*cos(#3)},{#5*sin(#3)})$) arc (#3:#4:#5); }

\hypersetup{
    colorlinks=true,
    linkcolor=blue,
    filecolor=blue,
    citecolor = black,      
    urlcolor=cyan,
    }

%%%%%%%%%%% Snippets %%%%%%%%%%%%%%%%
\newcommand*\widefbox[1]{\fbox{\hspace{2em}#1\hspace{2em}}}
\newcommand{\xint}{\int_{x_1}^{x_2}}
\newcommand{\mw}{\sqrt{m\omega}}
\newcommand{\de}{\delta}
\newcommand{\dde}{\dot{\delta}}
\newcommand{\di}{\delta_i}
\newcommand{\ddi}{\dot{\delta_i}}
\newcommand{\dddi}{\ddot{\delta_i}}
\newcommand{\dipl}{\delta_{i+1}}
\newcommand{\dimi}{\delta_{i-1}}
\newcommand{\ddt}[1]{\frac{{d} #1}{dt}}
\newcommand{\ddtt}[1]{\frac{d^2 #1}{dt^2}}
\newcommand{\ddx}[1]{\frac{d #1}{dx}}
\newcommand{\ddxx}[1]{\frac{d^2 #1}{dx^2}}
\newcommand{\eps}{\epsilon}
\newcommand{\del}[2]{\frac{\partial #1}{\partial #2}}
\newcommand{\deltwo}[2]{\frac{\partial^2 #1}{\partial #2^2}}
\newcommand{\lam}{\lambda}
\newcommand{\Lam}{\Lambda}
\newcommand{\sig}{\sigma}
\newcommand{\Sig}{\Sigma}
\newcommand{\half}{\frac{1}{2}}
\newcommand{\munu}{{\mu\nu}}
\newcommand{\thalf}{\tfrac{1}{2}}
\renewcommand{\div}{\nabla\cdot}
\renewcommand{\curl}{\nabla\times}

\DeclareMathOperator{\orb}{Orb}
\DeclareMathOperator{\stab}{Stab}
\DeclareMathOperator{\adj}{adj}
\DeclareMathOperator{\ccl}{ccl}
\let\var\relax
\DeclareMathOperator{\var}{Var}
\DeclareMathOperator{\cov}{Cov}
\DeclareMathOperator{\corr}{Corr}
\DeclareMathOperator{\Markov}{Markov}
\DeclareMathOperator{\nullity}{nullity}

\newcommand{\bfA}{{\bf A}}
\newcommand{\bfB}{{\bf B}}
\newcommand{\bfC}{{\bf C}}
\newcommand{\bfD}{{\bf D}}
\newcommand{\bfE}{{\bf E}}
\newcommand{\bfF}{{\bf F}}
\newcommand{\bfG}{{\bf G}}
\newcommand{\bfH}{{\bf H}}
\newcommand{\bfI}{{\bf I}}
\newcommand{\bfJ}{{\bf J}}
\newcommand{\bfK}{{\bf K}}
\newcommand{\bfL}{{\bf L}}
\newcommand{\bfM}{{\bf M}}
\newcommand{\bfN}{{\bf N}}
\newcommand{\bfO}{{\bf O}}
\newcommand{\bfP}{{\bf P}}
\newcommand{\bfQ}{{\bf Q}}
\newcommand{\bfR}{{\bf R}}
\newcommand{\bfS}{{\bf S}}
\newcommand{\bfT}{{\bf T}}
\newcommand{\bfU}{{\bf U}}
\newcommand{\bfV}{{\bf V}}
\newcommand{\bfW}{{\bf W}}
\newcommand{\bfX}{{\bf X}}
\newcommand{\bfY}{{\bf Y}}
\newcommand{\bfZ}{{\bf Z}}

\newcommand{\bfa}{{\bf a}}
\newcommand{\bfb}{{\bf b}}
\newcommand{\bfc}{{\bf c}}
\newcommand{\bfd}{{\bf d}}
\newcommand{\bfe}{{\bf e}}
\newcommand{\bff}{{\bf f}}
\newcommand{\bfg}{{\bf g}}
\newcommand{\bfh}{{\bf h}}
\newcommand{\bfi}{{\bf i}}
\newcommand{\bfj}{{\bf j}}
\newcommand{\bfk}{{\bf k}}
\newcommand{\bfl}{{\bf l}}
\newcommand{\bfm}{{\bf m}}
\newcommand{\bfn}{{\bf n}}
\newcommand{\bfo}{{\bf o}}
\newcommand{\bfp}{{\bf p}}
\newcommand{\bfq}{{\bf q}}
\newcommand{\bfr}{{\bf r}}
\newcommand{\bfs}{{\bf s}}
\newcommand{\bft}{{\bf t}}
\newcommand{\bfu}{{\bf u}}
\newcommand{\bfv}{{\bf v}}
\newcommand{\bfw}{{\bf w}}
\newcommand{\bfx}{{\bf x}}
\newcommand{\bfy}{{\bf y}}
\newcommand{\bfz}{{\bf z}}

\newcommand{\mcA}{{\mathcal{A}}}
\newcommand{\mcB}{{\mathcal{B}}}
\newcommand{\mcC}{{\mathcal{C}}}
\newcommand{\mcD}{{\mathcal{D}}}
\newcommand{\mcE}{{\mathcal{E}}}
\newcommand{\mcF}{{\mathcal{F}}}
\newcommand{\mcG}{{\mathcal{G}}}
\newcommand{\mcH}{{\mathcal{H}}}
\newcommand{\mcI}{{\mathcal{I}}}
\newcommand{\mcJ}{{\mathcal{J}}}
\newcommand{\mcK}{{\mathcal{K}}}
\newcommand{\mcL}{{\mathcal{L}}}
\newcommand{\mcM}{{\mathcal{M}}}
\newcommand{\mcN}{{\mathcal{N}}}
\newcommand{\mcO}{{\mathcal{O}}}
\newcommand{\mcP}{{\mathcal{P}}}
\newcommand{\mcQ}{{\mathcal{Q}}}
\newcommand{\mcR}{{\mathcal{R}}}
\newcommand{\mcS}{{\mathcal{S}}}
\newcommand{\mcT}{{\mathcal{T}}}
\newcommand{\mcU}{{\mathcal{U}}}
\newcommand{\mcV}{{\mathcal{V}}}
\newcommand{\mcW}{{\mathcal{W}}}
\newcommand{\mcX}{{\mathcal{X}}}
\newcommand{\mcY}{{\mathcal{Y}}}
\newcommand{\mcZ}{{\mathcal{Z}}}

\newcommand{\bbA}{{\mathbb{A}}}
\newcommand{\bbB}{{\mathbb{B}}}
\newcommand{\bbC}{{\mathbb{C}}}
\newcommand{\bbD}{{\mathbb{D}}}
\newcommand{\bbE}{{\mathbb{E}}}
\newcommand{\bbF}{{\mathbb{F}}}
\newcommand{\bbG}{{\mathbb{G}}}
\newcommand{\bbH}{{\mathbb{H}}}
\newcommand{\bbI}{{\mathbb{I}}}
\newcommand{\bbJ}{{\mathbb{J}}}
\newcommand{\bbK}{{\mathbb{K}}}
\newcommand{\bbL}{{\mathbb{L}}}
\newcommand{\bbM}{{\mathbb{M}}}
\newcommand{\bbN}{{\mathbb{N}}}
\newcommand{\bbO}{{\mathbb{O}}}
\newcommand{\bbP}{{\mathbb{P}}}
\newcommand{\bbQ}{{\mathbb{Q}}}
\newcommand{\bbR}{{\mathbb{R}}}
\newcommand{\bbS}{{\mathbb{S}}}
\newcommand{\bbT}{{\mathbb{T}}}
\newcommand{\bbU}{{\mathbb{U}}}
\newcommand{\bbV}{{\mathbb{V}}}
\newcommand{\bbW}{{\mathbb{W}}}
\newcommand{\bbX}{{\mathbb{X}}}
\newcommand{\bbY}{{\mathbb{Y}}}
\newcommand{\bbZ}{{\mathbb{Z}}}

\newcommand{\mfa}{{\mathfrak{a}}}
\newcommand{\mfb}{{\mathfrak{b}}}
\newcommand{\mfc}{{\mathfrak{c}}}
\newcommand{\mfd}{{\mathfrak{d}}}
\newcommand{\mfe}{{\mathfrak{e}}}
\newcommand{\mff}{{\mathfrak{f}}}
\newcommand{\mfg}{{\mathfrak{g}}}
\newcommand{\mfh}{{\mathfrak{h}}}
\newcommand{\mfi}{{\mathfrak{i}}}
\newcommand{\mfj}{{\mathfrak{j}}}
\newcommand{\mfk}{{\mathfrak{k}}}
\newcommand{\mfl}{{\mathfrak{l}}}
\newcommand{\mfm}{{\mathfrak{m}}}
\newcommand{\mfn}{{\mathfrak{n}}}
\newcommand{\mfo}{{\mathfrak{o}}}
\newcommand{\mfp}{{\mathfrak{p}}}
\newcommand{\mfq}{{\mathfrak{q}}}
\newcommand{\mfr}{{\mathfrak{r}}}
\newcommand{\mfs}{{\mathfrak{s}}}
\newcommand{\mft}{{\mathfrak{t}}}
\newcommand{\mfu}{{\mathfrak{u}}}
\newcommand{\mfv}{{\mathfrak{v}}}
\newcommand{\mfw}{{\mathfrak{w}}}
\newcommand{\mfx}{{\mathfrak{x}}}
\newcommand{\mfy}{{\mathfrak{y}}}
\newcommand{\mfz}{{\mathfrak{z}}}

\newcommand{\mfA}{{\mathfrak{A}}}
\newcommand{\mfB}{{\mathfrak{B}}}
\newcommand{\mfC}{{\mathfrak{C}}}
\newcommand{\mfD}{{\mathfrak{D}}}
\newcommand{\mfE}{{\mathfrak{E}}}
\newcommand{\mfF}{{\mathfrak{F}}}
\newcommand{\mfG}{{\mathfrak{G}}}
\newcommand{\mfH}{{\mathfrak{H}}}
\newcommand{\mfI}{{\mathfrak{I}}}
\newcommand{\mfJ}{{\mathfrak{J}}}
\newcommand{\mfK}{{\mathfrak{K}}}
\newcommand{\mfL}{{\mathfrak{L}}}
\newcommand{\mfM}{{\mathfrak{M}}}
\newcommand{\mfN}{{\mathfrak{N}}}
\newcommand{\mfO}{{\mathfrak{O}}}
\newcommand{\mfP}{{\mathfrak{P}}}
\newcommand{\mfQ}{{\mathfrak{Q}}}
\newcommand{\mfR}{{\mathfrak{R}}}
\newcommand{\mfS}{{\mathfrak{S}}}
\newcommand{\mfT}{{\mathfrak{T}}}
\newcommand{\mfU}{{\mathfrak{U}}}
\newcommand{\mfV}{{\mathfrak{V}}}
\newcommand{\mfW}{{\mathfrak{W}}}
\newcommand{\mfX}{{\mathfrak{X}}}
\newcommand{\mfY}{{\mathfrak{Y}}}
\newcommand{\mfZ}{{\mathfrak{Z}}}

\newcommand{\rma}{\mathrm{a}}
\newcommand{\rmb}{\mathrm{b}}
\newcommand{\rmc}{\mathrm{c}}
\newcommand{\rmd}{\mathrm{d}}
\renewcommand{\dd}{\,\mathrm{d}}
\newcommand{\rme}{\mathrm{e}}
\newcommand{\rmf}{\mathrm{f}}
\newcommand{\rmg}{\mathrm{g}}
\newcommand{\rmh}{\mathrm{h}}
\newcommand{\rmi}{\mathrm{i}}
\newcommand{\rmj}{\mathrm{j}}
\newcommand{\rmk}{\mathrm{k}}
\newcommand{\rml}{\mathrm{l}}
\newcommand{\rmm}{\mathrm{m}}
\newcommand{\rmn}{\mathrm{n}}
\newcommand{\rmo}{\mathrm{o}}
\newcommand{\rmp}{\mathrm{p}}
\newcommand{\rmq}{\mathrm{q}}
\newcommand{\rmr}{\mathrm{r}}
\newcommand{\rms}{\mathrm{s}}
\newcommand{\rmt}{\mathrm{t}}
\newcommand{\rmu}{\mathrm{u}}
\newcommand{\rmv}{\mathrm{v}}
\newcommand{\rmw}{\mathrm{w}}
\newcommand{\rmx}{\mathrm{x}}
\newcommand{\rmy}{\mathrm{y}}
\newcommand{\rmz}{\mathrm{z}}
\newcommand{\rmA}{\mathrm{A}}
\newcommand{\rmB}{\mathrm{B}}
\newcommand{\rmC}{\mathrm{C}}
\newcommand{\rmD}{\mathrm{D}}
\newcommand{\rmE}{\mathrm{E}}
\newcommand{\rmF}{\mathrm{F}}
\newcommand{\rmG}{\mathrm{G}}
\newcommand{\rmH}{\mathrm{H}}
\newcommand{\rmI}{\mathrm{I}}
\newcommand{\rmJ}{\mathrm{J}}
\newcommand{\rmK}{\mathrm{K}}
\newcommand{\rmL}{\mathrm{L}}
\newcommand{\rmM}{\mathrm{M}}
\newcommand{\rmN}{\mathrm{N}}
\newcommand{\rmO}{\mathrm{O}}
\newcommand{\rmP}{\mathrm{P}}
\newcommand{\rmQ}{\mathrm{Q}}
\newcommand{\rmR}{\mathrm{R}}
\newcommand{\rmS}{\mathrm{S}}
\newcommand{\rmT}{\mathrm{T}}
\newcommand{\rmU}{\mathrm{U}}
\newcommand{\rmV}{\mathrm{V}}
\newcommand{\rmW}{\mathrm{W}}
\newcommand{\rmX}{\mathrm{X}}
\newcommand{\rmY}{\mathrm{Y}}
\newcommand{\rmZ}{\mathrm{Z}}

\newcommand{\GL}{\mathrm{GL}}
\newcommand{\Or}{\mathrm{O}}
\newcommand{\PGL}{\mathrm{PGL}}
\newcommand{\PSL}{\mathrm{PSL}}
\newcommand{\PSO}{\mathrm{PSO}}
\newcommand{\PSU}{\mathrm{PSU}}
\newcommand{\SL}{\mathrm{SL}}
\newcommand{\SO}{\mathrm{SO}}
\newcommand{\Spin}{\mathrm{Spin}}
\newcommand{\Sp}{\mathrm{Sp}}
\newcommand{\SU}{\mathrm{SU}}
\newcommand{\Mat}{\mathrm{Mat}}

% Some common notations

\renewcommand{\v}{\mathbf{v}}
\newcommand{\w}{\mathbf{w}}
\renewcommand{\u}{\mathbf{u}}

% Matrix algebras
\newcommand{\gl}{\mathfrak{gl}}
\newcommand{\ort}{\mathfrak{o}}
\newcommand{\so}{\mathfrak{so}}
\newcommand{\su}{\mathfrak{su}}
\newcommand{\uu}{\mathfrak{u}}
\renewcommand{\sl}{\mathfrak{sl}}
\newcommand{\inner}[1]{\left\langle{#1}\right\rangle}
\DeclareMathOperator{\spn}{span}

\newcommand{\mobius}{{M\"{o}bius }}

\renewcommand{\ge}{\geqslant}
\renewcommand{\le}{\leqslant}
\renewcommand{\geq}{\geqslant}
\renewcommand{\leq}{\leqslant}
\renewcommand{\restriction}{\mathord{\upharpoonright}}

\newcommand\independent{\protect\mathpalette{\protect\independenT}{\perp}}
\def\independenT#1#2{\mathrel{\rlap{$#1#2$}\mkern2mu{#1#2}}}

\setlength{\parindent}{0pt}
% \setlength{\parskip}{\baselineskip}
\newcommand{\incfig}[1]{%
    \def\svgwidth{0.4\columnwidth}
    \import{./figures/}{#1.pdf_tex}
}
%%%%%%%%%%%%%%%%%%%%%%%%%%%%%%%%%%%%%

\usepackage[T1]{fontenc}
\usepackage{lmodern,mathrsfs}

%%%%%%%boxed enviroment for final layout%%%%%%%%%%%%%

\newtheoremstyle{mystyle}%
  {}%
  {}%
  {}%
  {}%
  {\sffamily\bfseries}%
  {.}%
  { }%
  {}%

% \renewenvironment{proof}{{\sffamily\bfseries Proof. }}{\qed}

\theoremstyle{mystyle}{
  \newtheorem{theorem}{Theorem}[section]
  \newtheorem{lemma}[theorem]{Lemma}
  \newtheorem{proposition}[theorem]{Proposition}
  \newtheorem{corollary}[theorem]{Corollary}
  \newtheorem{problem}[theorem]{Problem}
  \newtheorem*{claim}{Claim}
  \newtheorem*{slemma}{Lemma}
  \newtheorem*{sprop}{Proposition}
  \newtheorem*{notation}{Notation}

  \newtheorem{inquestion}{Question}
  \newtheorem*{sque}{Question}

  \newtheorem{definition}{Definition}[section]
  \newtheorem{conjecture}{Conjecture}[section]
  \newtheorem{example}{Example}[section]
  \newtheorem*{law}{Law}

  \newtheorem*{remark}{Remark}
  \newtheorem*{note}{Note}
}

\newenvironment{question}[1]
{\renewcommand\theinquestion{#1}\inquestion}
{\endinquestion}

\theoremstyle{definition}{
    \newtheorem*{exercise}{Exercise}}

\tcolorboxenvironment{definition}{
  boxrule=0pt,
  boxsep=2pt,
  colback={White!90!Cerulean},
  enhanced jigsaw, 
  borderline west={2pt}{0pt}{Cerulean},
  sharp corners,
  before skip=10pt,
  after skip=10pt,
  breakable,
  % parbox=false,
}

\tcolorboxenvironment{notation}{
  boxrule=0pt,
  boxsep=2pt,
  colback={White!90!Cerulean},
  enhanced jigsaw, 
  borderline west={2pt}{0pt}{Cerulean},
  sharp corners,
  before skip=10pt,
  after skip=10pt,
  breakable,
  % parbox=false,
}

\tcolorboxenvironment{proposition}{
  boxrule=0pt,
  boxsep=2pt,
  colback={White!90!Yellow},
  enhanced jigsaw, 
  borderline west={2pt}{0pt}{Yellow},
  sharp corners,
  before skip=10pt,
  after skip=10pt,
  breakable,
  % parbox=false,
}

\tcolorboxenvironment{sprop}{
  boxrule=0pt,
  boxsep=2pt,
  colback={White!90!Yellow},
  enhanced jigsaw, 
  borderline west={2pt}{0pt}{Yellow},
  sharp corners,
  before skip=10pt,
  after skip=10pt,
  breakable,
  % parbox=false,
}

\tcolorboxenvironment{theorem}{
  boxrule=0pt,
  boxsep=2pt,
  colback={White!90!Dandelion},
  enhanced jigsaw, 
  borderline west={2pt}{0pt}{Dandelion},
  sharp corners,
  before skip=10pt,
  after skip=10pt,
  breakable,
  % parbox=false,
}

\tcolorboxenvironment{lemma}{
  boxrule=0pt,
  boxsep=2pt,
  blanker,
  borderline west={2pt}{0pt}{Red},
  before skip=10pt,
  after skip=10pt,
  sharp corners,
  left=12pt,
  right=12pt,
  breakable,
  % parbox=false,
}

\tcolorboxenvironment{corollary}{
  boxrule=0pt,
  boxsep=2pt,
  blanker,
  borderline west={2pt}{0pt}{ForestGreen},
  before skip=10pt,
  after skip=10pt,
  sharp corners,
  left=12pt,
  right=12pt,
  breakable,
  % parbox=false,
}

\tcolorboxenvironment{proof}{
  boxrule=0pt,
  boxsep=2pt,
  blanker,
  borderline west={2pt}{0pt}{NavyBlue!80!white},
  before skip=10pt,
  after skip=10pt,
  left=12pt,
  right=12pt,
  breakable,
  % parbox=false,
}

\tcolorboxenvironment{remark}{
  boxrule=0pt,
  boxsep=2pt,
  blanker,
  borderline west={2pt}{0pt}{Green},
  before skip=10pt,
  after skip=10pt,
  left=12pt,
  right=12pt,
  breakable,
  % parbox=false,
}

\tcolorboxenvironment{note}{
  boxrule=0pt,
  boxsep=2pt,
  blanker,
  borderline west={2pt}{0pt}{PineGreen},
  before skip=10pt,
  after skip=10pt,
  left=12pt,
  right=12pt,
  breakable,
  % parbox=false,
}

\tcolorboxenvironment{example}{
  boxrule=0pt,
  boxsep=2pt,
  blanker,
  borderline west={2pt}{0pt}{Black},
  sharp corners,
  before skip=10pt,
  after skip=10pt,
  left=12pt,
  right=12pt,
  breakable,
  % parbox=false,
}

\titleformat*{\section}{\Large\bfseries\sffamily}
\titleformat*{\subsection}{\large\bfseries\sffamily}
\titleformat*{\subsubsection}{\bfseries\sffamily}
\titleformat*{\paragraph}{\bfseries\sffamily}

%%%%%%%%%%%%%%%%%%%%%%%%%%%%%%%%%%%%%%%%%%%%%%%%%%%%

\title{\textbf{\sffamily\triposcourse{} Notes}}
% \usepackage[T1]{fontenc}
\usepackage{crimson}

\theoremstyle{plain}

\theoremstyle{definition}
\newtheorem{theorem}{Theorem}[section]
\newtheorem{lemma}[theorem]{Lemma}
\newtheorem{proposition}[theorem]{Proposition}
\newtheorem{corollary}[theorem]{Corollary}
\newtheorem{problem}[theorem]{Problem}
\newtheorem*{claim}{Claim}
\newtheorem*{slemma}{Lemma}
\newtheorem*{sprop}{Proposition}
\newtheorem*{notation}{Notation}
\newtheorem*{exercise}{Exercise}

\newtheorem{inquestion}{Question}
\newtheorem*{sque}{Question}
\newenvironment{question}[1]
  {\renewcommand\theinquestion{#1}\inquestion}
  {\endinquestion}

\newtheorem{definition}{Definition}[section]
\newtheorem{conjecture}{Conjecture}[section]
\newtheorem{example}{Example}[section]
\newtheorem*{law}{Law}

\theoremstyle{remark}
\newtheorem*{remark}{Remark}
\newtheorem*{note}{Note}

\title{\textbf{\triposcourse{} Notes}}
% \theoremstyle{plain}{
  \newtheorem{theorem}{Theorem}[section]
  \newtheorem{lemma}[theorem]{Lemma}
  \newtheorem{proposition}[theorem]{Proposition}
  \newtheorem{corollary}[theorem]{Corollary}
  \newtheorem*{claim}{Claim}
  \newtheorem*{slemma}{Lemma}
  \newtheorem*{sprop}{Proposition}
  \newtheorem{conjecture}{Conjecture}[section]
  \newtheorem*{law}{Law}
  \newtheorem{inquestion}{Question}
  \newtheorem*{sque}{Question}
}

\theoremstyle{definition}{
  \newtheorem{method}[theorem]{Method}
  \newtheorem{definition}{Definition}[section]
  \newtheorem{example}{Example}[section]
  \newtheorem*{notation}{Notation}
  \newtheorem*{exercise}{Exercise}
}

\theoremstyle{remark}{
  \newtheorem{remark}[theorem]{Remark}
  \newtheorem*{note}{Note}
}

\newenvironment{question}[1]
{\renewcommand\theinquestion{#1}\inquestion}
{\endinquestion}

\title{\textbf{\sffamily\triposcourse{} Notes}}

%layout full
% \geometry{%
%   a4paper,
%   lmargin=2cm,
%   rmargin=2.5cm,
%   tmargin=3.5cm,
%   bmargin=2.5cm,
%   footskip=12pt,
%   headheight=24pt}
% layout trim
% \geometry{
% papersize={379pt, 542pt},
% textwidth=345pt,
% textheight=443pt,
% left=17pt,
% top=54pt,
% right=17pt
% }
% layout a5
\geometry{%
  a5paper,
  lmargin=1cm,
  rmargin=1cm,
  tmargin=2.5cm,
  bmargin=1.5cm,
  footskip=15pt,
  headheight=24pt}
\pagestyle{fancy}
\rhead{{\triposcourse{}}}
\author{jt775}
\AddToHook{cmd/section/before}{\clearpage}

\counterwithin{equation}{section}
\graphicspath{ {./images/} }
\pgfplotsset{compat=1.17}
\begin{document}
\maketitle
\tableofcontents
\clearpage

\section{Surfaces}
\subsection{Topological surfaces}
Let's start with a definition.
\begin{definition}
    A \textbf{topological surface} is a topological space $ \Sigma $ such that 
    \begin{enumerate}[(a)]
        \item $ \forall \rho\in \Sigma $, there is an open neighbourhood $\rho\in U \subset \Sigma$ such that $U$ is homeomorphic to $\mathbb{R}^2$, or a disc $D^2 \subset \mathbb{R}^{2}$, with its usual Euclidean topology. 
        \item $\Sigma$ is Hausdorff and second countable. 
    \end{enumerate}
\end{definition}
\begin{remark}
    \begin{enumerate}[(1)]
        \item Write $ \cong $ for homeomorphic, $ \mathbb{R}^{2} \cong D^2(0,1) = \{x\in \mathbb{R}^{2}: \left\| x \right\|<1\} $

        \item A space $X$ is Hausdorff if for $p\neq q$ in $X$, there exists disjoint open sets $U\ni p, V\ni q$ in $X$. 
        A space is second countable if it has a countable base, i.e. there exists $ \{U_i\}_{i\in \mathbb{N}} $ open sets such that every open set is a union of some $U_i$. (a) is the point and (b) is for technical honesty.
             
        \item If $X$ is Hausdorff or second countable, so are subspaces of $X$. Euclidean space has these properties. To see it is second countable, consider open balls $B(c,r)$ with $c\in \mathbb{Q}^n$ and $r\in \mathbb{Q}_+$. 
    \end{enumerate}
\end{remark}

\begin{example}
    \begin{enumerate}
        \item $ \mathbb{R}^{2}$ the plane. 
        \item Any open set in $ \mathbb{R}^{2}$, i.e. $ \mathbb{R}^{2}\setminus Z $ where $Z$ is closed, e.g. $ Z = \{0\} $ or $ Z = \{(0,0) \cup (0,\frac{1}{n}), n=1,2,\dots\} $. 
    \end{enumerate}
\end{example}

\begin{example}
    Let \( f \colon \mathbb R^2 \to \mathbb R \) be a continuous function.
	The graph of \( f \), denoted \( \Gamma_f \), is defined by
	\[
		\Gamma_f = \qty{(x,y,f(x,y)) \colon (x,y) \in \mathbb R^2}
	\]
	with the subspace topology when embedded in \( \mathbb R^3 \).

	Recall that a product topology \( X \times Y \) has the feature that \( f \colon Z \to X \times Y \) is continuous if and only if \( \pi_x \circ f \colon Z \to X \) and \( \pi_y \circ f \colon Z \to Y \) are continuous.

	Hence, any graph \( \Gamma \subseteq X \times Y \) is homeomorphic to \( X \) if \( f \) is continuous.
	Indeed, the projection \( \pi_x \) projects each point in the graph onto the domain.
	The function \( s \colon x \mapsto (x,f(x)) \) is continuous by the above.
	In particular, in our case, the graph \( \Gamma_f \) is homeomorphic to \( \mathbb R^2 \), which we know is a topological surface.
\end{example}
\begin{remark}
	As a topological surface, \( \Gamma_f \) is independent of the function \( f \).
	However, we will later introduce more ways to describe topological spaces that will ascribe new properties to \( \Gamma_f \) which do depend on \( f \).
\end{remark}
\begin{example}
	The sphere:
	\[
		S^2 = \qty{(x,y,z) \in \mathbb R^3 \colon x^2 + y^2 + z^2 = 1}
	\]
	is a topological surface, when using the subspace topology in \( \mathbb R^3 \).
	Consider the stereographic projection of \( S^2 \) onto \( \mathbb R^2 \) from the north pole \( (0,0,1) \).
	The projection satisfies \( \pi_+ \colon S^2 \setminus \qty{(0,0,1)} \) and
	\[
		(x,y,z) \mapsto \qty(\frac{x}{1-z}, \frac{y}{1-z})
	\]
	Certainly, \( \pi_+ \) is continuous, since we do not consider the point \( (0,0,1) \) to be in its domain.
	The inverse map is given by
	\[
		(u,v) \mapsto \qty(\frac{2u}{u^2+v^2+1}, \frac{2v}{u^2+v^2+1}, \frac{u^2+v^2-1}{u^2+v^2+1})
	\]
	This is also a continuous function.
	Hence \( \pi_+ \) is a homeomorphism.
	Similarly, we can construct the stereographic projection from the south pole, \( \pi_- \) defined by 
    \[
        (x,y,z) \mapsto \left( \frac{x}{1+z},\frac{y}{1+z} \right). 
    \]
	This is a homeomorphism.
	Hence, every point in \( S^2 \) lies either in the domain of \( \pi_+ \) or \( \pi_- \), and hence sits in an open set \( S^2 \setminus \qty{(0,0,1)} \) or \( S^2 \setminus \qty{(0,0,-1)} \) which is homeomorphic to \( \mathbb R^2 \).
\end{example}
\begin{note}
	\( S^2 \) is compact by the Heine-Borel theorem, that is, it is a closed bounded set in \( \mathbb R^3 \).
\end{note}

\begin{example}
	The real projective plane is a topological surface.
	The group \( \mathbb Z / 2 \) acts on \( S^2 \) by homeomorphisms via the \textbf{antipodal map} \( a \colon S^2 \to S^2 \), mapping \( x \mapsto -x \).
	There exists a homeomorphism \( \mathbb Z / 2 \) to the group \( \mathrm{Homeo}(S^2) \) of homeomorphisms of \( S^2 \), by mapping \( 1 + \mathbb Z \mapsto a \).

	We now define the real projective plane to be the quotient of \( S^2 \) given by identifying every point \( x \) with its image \( -x \) under \( a \).
	\[
		\mathbb R \mathbb P^2 = \faktor{S^2}{\mathbb Z/2} = \faktor{S^2}{\sim};\quad x \sim a(x)
	\]
	\begin{lemma}
		\( \mathbb R \mathbb P^2 \) bijects with the set of straight lines in \( \mathbb R^3 \) through the origin.
	\end{lemma}
	\begin{proof}
		Any line through the origin intersects \( S^2 \) exactly in a pair of antipodal points \( x, -x \).
		Similarly, pairs of antipodal points uniquely define a line through the origin.
	\end{proof}
	\begin{lemma}
		\( \mathbb R \mathbb P^2 \) is a topological surface with the quotient topology.
	\end{lemma}
	\begin{proof}
		First check that \( \mathbb R \mathbb P^2 \) is Hausdorff since it is constructed by a quotient, not a subspace.
		If \( X \) is a space and \( q \colon X \to Y \) is a quotient map, then by definition \( V \subset Y \) is open if and only if \( q^{-1}(V) \subset X \) is open.
		If \( [p] \neq [q] \in \mathbb R \mathbb P^2 \), then \( \pm p, \pm q \in S^2 \) are distinct antipodal pairs.
		We can therefore construct distinct open discs around \( p, q \) in \( S^2 \), and their antipodal images.
		These uniquely define open neighbourhoods of \( [p], [q] \), which are disjoint.

		Similarly, we can check that \( \mathbb R \mathbb P^2 \) is second countable.
		We know that \( S^2 \) is second countable, so let \( \mathcal U \) be a countable base for the topology on \( S^2 \).
		Without loss of generality, we can assert that for all sets \( U \in \mathcal U \), we have \( -U \in \mathcal U \).
		Let \( \overline{\mathcal U} \) be the family of open sets in \( \mathbb R \mathbb P^2 \) of the form \( q(U) \cup q(-U) \) for \( U \in \mathcal U \), where \( q \) is the quotient map.
		Now, if \( V \subseteq \mathbb R \mathbb P^2 \) is open, then by definition \( q^{-1}(V) \) is open in \( S^2 \) hence \( q^{-1}(V) \) contains some \( U \in \mathcal U \) and hence contains \( U \cup (-U) \).
		Hence \( \overline{\mathcal U} \) is a countable base for the quotient topology on \( \mathbb R \mathbb P^2 \).

		Finally, let \( p \in S^2 \) and \( [p] \in \mathbb R \mathbb P^2 \) its image.
		Let \( \overline D \) be a small (contained in an open hemisphere) closed disc, which is a neighbourhood of \( p \in S^2 \).
		The quotient map restricted to \( \overline D \), written \( \eval{q}_{\overline D} \colon \overline D \to q(\overline D) \subset \mathbb R \mathbb P^2 \), is a continuous function from a compact space to a Hausdorff space.
		Further, \( q \) is injective on \( \overline D \) since the disc was contained entirely in a single hemisphere.
		The topological inverse function theorem (TIFT) states that a continuous bijection from a compact space to a Hausdorff space is a homeomorphism.
		So \( \eval{q}_{\overline D} \) is a homeomorphism from \( \overline D \) to \( q(\overline D) \).
		This then induces the homeomorphism \( \eval{q}_{D} \) from the open disc \( D = {\overline D}^\circ \) to \( q(D) \).
		So by construction, \( [p] \in q(D) \); it has an open neighbourhood in \( \mathbb R \mathbb P^2 \) which is homeomorphic to an open disc, concluding the proof.
	\end{proof}
\end{example}

\begin{example}
	Let \( S^1 \) be the unit circle in \( \mathbb C \), and define the torus to be the product space \( S^1 \times S^1 \), with the subspace topology from \( \mathbb C^2 \) (identical to the product topology).
	\begin{lemma}
		The torus $ S^1 \times S^1 $ is a topological surface.
	\end{lemma}
	\begin{proof}
		Consider the map \( e \colon \mathbb R^2 \to S^1 \times S^1 \subset \mathbb{C} \times \mathbb{C} \) defined by
		\[
			(s,t) \mapsto \qty(e^{2\pi i s}, e^{2 \pi i t})
		\]
		That this induces a map \( \hat e \) from \( {\mathbb R^2}/{\mathbb Z^2} \), since \( e \) is constant under translations by \( \mathbb Z^2 \).
		\begin{center}
			\begin{tikzcd}
				\mathbb R^2 \arrow[d, "q"'] \arrow[r, "e"]             & S^1 \times S^1 \\
				{\mathbb R^2}/{\mathbb Z^2} \arrow[ru,dashed, "\hat e"] &
			\end{tikzcd}
		\end{center}
		Under the quotient topology given by the quotient map \( q \), \( {\mathbb R^2}/{\mathbb Z^2} \) is a topological space.
		The map \( [0,1]^2 \to \mathbb R^2 \to {\mathbb R^2}/{\mathbb Z^2} \) is surjective, so \( {\mathbb R^2}/{\mathbb Z^2} \) is compact.
		So \( \hat e \) is a continuous map from a compact space to a Hausdorff space, and \( \hat e \) is bijective, so by TIFT \( \hat e \) is a homeomorphism.

		Note that we already have that \( S^1 \times S^1 \) is compact, Hausdorff and second countable (as a closed and bounded set in \( \mathbb C^2 \)), so it suffices to show it is locally homeomorphic to \( \mathbb R^2 \).

		As for the case of $S^2\to \mathbb{R}\mathbb{P}^2$, take $p\in \mathbb{R}^2, [p]= q(p)\in S^1\times S^1 $ and a small closed disc $ \bar{D}(p)\in \mathbb{R}\mathbb{P}^2 $ such that 
		\[
			\forall	 (m,n)\neq (0,0)\in \mathbb{Z}^2, \overline{D}(p) \cap (\overline{D}(p)+(m,n)) = \varnothing. 
		\]
		Hence \( \eval{e}_{\overline D(p)} \) is injective and \( \eval{q}_{\overline D(p)} \) is injective.
		Now, restricting to the open disc as before, we can find an open disc neighbourhood of \( [p] \).
		Since \( [p] \) was chosen arbitrarily, \( S^1 \times S^1 \) is a topological surface.
	\end{proof}
	Another view point is that $ \mathbb{R}^{2}/\mathbb{Z}^2 $ is also given by imposing on $[0,1]^2$ the equivalence relation that 
\[
	(x,0) \sim (x,1),\ \forall 0\le x\le 1,\quad (0,y) \sim (1,y),\ \forall 0\le y\le 1.
\]
\begin{center}
	\begin{tikzpicture}
	\draw [red, ->>-=0.63] (0, 0) -- +(2, 0);
	\draw [red, ->>-=0.63] (0, 2) -- +(2, 0);

	\draw [blue, ->-=0.58] (0, 0) -- +(0, 2);
	\draw [blue, ->-=0.58] (2, 0) -- +(0, 2);
	\node[anchor=east] at (0, 0.5) {$ (0,y) $};
	\node[dot=3pt] at (0,0.5) {};
	\node[anchor=west] at (2, 0.5) {$ (1,y) $};
	\node[dot=3pt] at (2,0.5) {};
	\node[anchor=north] at (1.5, 0) {$ (x,0) $};
	\node[dot=3pt] at (1.5,0) {};
	\node[anchor=south] at (1.5, 2) {$ (x,1) $};
	\node[dot=3pt] at (1.5, 2) {};
	\node at (3.4,1) {$\cong$};
	\begin{scope}[shift={(6,1)}]
	\draw (0,0) ellipse (2 and 1.12);
	\path[rounded corners=24pt] (-.9,0)--(0,.6)--(.9,0) (-.9,0)--(0,-.56)--(.9,0);
	\draw[rounded corners=28pt] (-1.1,.1)--(0,-.6)--(1.1,.1);
	\draw[rounded corners=24pt] (-.9,0)--(0,.6)--(.9,0);
	\end{scope}
	\end{tikzpicture}
  \end{center}
\end{example}

\begin{example}
	Let \( P \) be a planar Euclidean polygon, with oriented edges.
	We will pair the edges, and without loss of generality we will assume that paired edges have the same Euclidean length.
	\begin{center}
		\begin{tikzpicture}
			\draw [red, ->>-=0.63] (0, 0) -- +(2, 0);
			\draw [red, ->>-=0.63] (0, 2) -- +(2, 0);
		
			\draw [blue, ->-=0.58] (0, 0) -- +(0, 2);
			\draw [blue, ->-=0.58] (2, 0) -- +(0, 2);
			\node[anchor=east] at (0, 1) {$ a $};
			\node[anchor=east] at (2, 1) {$ a^{-1} $};
			\node[anchor=north] at (1, 0) {$ b^{-1} $};
			\node[anchor=north] at (1, 2) {$ b $};

			\begin{scope}[shift={(3.3,0)}]
			\draw [red, ->>-=0.63] (2, 0) -- (0, 0);
			\draw [red, ->>-=0.63] (0, 2) -- +(2, 0);
		
			\draw [blue, ->-=0.58] (0, 0) -- +(0, 2);
			\draw [blue, ->-=0.58] (2, 0) -- +(0, 2);
			\node[anchor=east] at (0, 1) {$ a $};
			\node[anchor=east] at (2, 1) {$ a^{-1} $};
			\node[anchor=north] at (1, 0) {$ b $};
			\node[anchor=north] at (1, 2) {$ b $};
			\end{scope}
			\begin{scope}[shift={(7,0)}]
			\draw [red, ->-=0.58] (0,0) -- (1,0);
			\draw [red, ->-=0.58] (0,2) -- (1,2);
			\draw [blue, ->>-=0.71] (-0.5,1) -- (0,0);
			\draw [blue, ->>-=0.71] (1,2) -- (1.5,1);
			\draw [Green,-trig- = 0.63] (-0.5,1) -- (0,2);
			\draw [Green,-trig- = 0.63] (1.5,1) -- (1,0);
			\end{scope}
		\end{tikzpicture}
	\end{center}
	Assign letter names to each edge pair, and denote a polygon by the sequence of edges found when traversing in a clockwise orientation.
	The edge pair name is inverted if the edge is traversed in the reverse direction.
	Note the difference between the annotations on the first two shapes above, due to the reversed direction of the edge.

	If two edges \( e, \hat e \) are paired, this defines a unique Euclidean isometry from \( e \) to \( \hat e \) respecting the orientation, \( f_{e\hat e} \colon e \to \hat e \).
	The set of all such functions generate an equivalence relation on the polygon, identifying paired edges with each other.
	\begin{lemma}
		$P/\sim $ with quotient topology is a topological surface. 
	\end{lemma}
\end{example}

\end{document}
\usepackage[T1]{fontenc}
\usepackage{lmodern,mathrsfs}

%%%%%%%boxed enviroment for final layout%%%%%%%%%%%%%

\newtheoremstyle{mystyle}%
  {}%
  {}%
  {}%
  {}%
  {\bfseries\sffamily}%
  {.}%
  { }%
  {}%

% \renewenvironment{proof}{{\bfseries Proof. }}{\qed}

\mdfsetup{
  skipabove=0.8\baselineskip,
  skipbelow=0.3\baselineskip,
  % innertopmargin=1\baselineskip,
  innerbottommargin=8pt,
  hidealllines=true
}

% Definition
\declaretheoremstyle[
    headfont=\sffamily\bfseries,
    mdframed={
      backgroundcolor=White!90!Cerulean,
      linecolor=Cerulean,
      linewidth=1pt,
      topline=false,
      rightline=false,
      bottomline=false,
      leftline=true
    },
    headpunct={\\[3pt]},
    postheadspace={3pt}
]{thmdefinitionbox}

% Notation
\declaretheoremstyle[
    headfont=\sffamily\bfseries,
    mdframed={
      backgroundcolor=White!90!Cerulean,
      linecolor=Cerulean,
      linewidth=1pt,
      topline=false,
      rightline=false,
      bottomline=false,
      leftline=true
    },
    headpunct={\\[3pt]},
    postheadspace={3pt}
]{thmnotationbox}

% Proposition
\declaretheoremstyle[
    headfont=\sffamily\bfseries,
    mdframed={
      backgroundcolor=White!90!Yellow,
      linecolor=Yellow,
      linewidth=1pt,
      topline=false,
      rightline=false,
      bottomline=false,
      leftline=true
    },
    headpunct={\\[3pt]},
    postheadspace={3pt}
]{thmpropositionbox}

% Theorem
\declaretheoremstyle[
    headfont=\sffamily\bfseries,
    mdframed={
      backgroundcolor=White!90!Dandelion,
      linecolor=Dandelion,
      linewidth=1pt,
      topline=false,
      rightline=false,
      bottomline=false,
      leftline=true
    },
    headpunct={\\[3pt]},
    postheadspace={3pt}
]{thmtheorembox}

% Lemma
\declaretheoremstyle[
    headfont=\sffamily\bfseries,
    mdframed={
      linecolor=Red,
      linewidth=1pt,
      topline=false,
      rightline=false,
      bottomline=false,
      leftline=true
    },
    headpunct={\\[3pt]},
    postheadspace={3pt}
]{thmlemmabox}

% Corollary
\declaretheoremstyle[
    headfont=\sffamily\bfseries,
    mdframed={
      linecolor=ForestGreen,
      linewidth=1pt,
      topline=false,
      rightline=false,
      bottomline=false,
      leftline=true
    },
    headpunct={\\[3pt]},
    postheadspace={3pt}
]{thmcorollarybox}

% Proof
\mdfdefinestyle{mdproofbox}{
    skipabove=0.4\baselineskip,
    skipbelow=0.3\baselineskip,
    linecolor=NavyBlue!80!white,
    linewidth=1pt,
    topline=false,
    innerbottommargin=6pt,
    rightline=false,
    bottomline=false,
    leftline=true
}

% Remark
\declaretheoremstyle[
    headfont=\sffamily\bfseries,
    mdframed={
      linecolor=Green,
      linewidth=1pt,
      topline=false,
      rightline=false,
      bottomline=false,
      leftline=true
    },
    headpunct={\\[3pt]},
    postheadspace={3pt}
]{thmremarkbox}

% Note
\declaretheoremstyle[
    headfont=\sffamily\bfseries,
    mdframed={
      linecolor=PineGreen,
      linewidth=1pt,
      topline=false,
      rightline=false,
      bottomline=false,
      leftline=true
    },
    headpunct={\\[3pt]},
    postheadspace={3pt}
]{thmnotebox}

% Example
\declaretheoremstyle[
    headfont=\sffamily\bfseries,
    mdframed={
      linecolor=Black,
      linewidth=1pt,
      topline=false,
      rightline=false,
      bottomline=false,
      leftline=true
    },
    headpunct={\\[3pt]},
    postheadspace={3pt}
]{thmexamplebox}

\declaretheorem[style=thmtheorembox,name=Theorem,numberwithin=section]{theorem}
\declaretheorem[style=thmdefinitionbox,name=Definition,sibling=theorem]{definition}
\declaretheorem[style=thmnotationbox,name=Notation,sibling=theorem]{notation}
\declaretheorem[style=thmpropositionbox,name=Proposition,sibling=theorem]{proposition}
\declaretheorem[style=thmpropositionbox,name=Proposition,sibling=theorem,numbered=no]{sprop}
\declaretheorem[style=thmlemmabox,name=Lemma,sibling=theorem]{lemma}
\declaretheorem[style=thmcorollarybox,name=Corollary,sibling=theorem]{corollary}
\declaretheorem[style=thmremarkbox,name=Remark,sibling=theorem,numbered=no]{remark}
\declaretheorem[style=thmnotebox,name=Note,sibling=theorem,numbered=no]{note}
\declaretheorem[style=thmexamplebox,name=Example,sibling=theorem]{example}

\theoremstyle{mystyle}{
%   \newtheorem{theorem}{Theorem}[section]
%   \newtheorem{lemma}[theorem]{Lemma}
%   \newtheorem{proposition}[theorem]{Proposition}
%   \newtheorem{corollary}[theorem]{Corollary}
  \newtheorem{problem}[theorem]{Problem}
  \newtheorem*{claim}{Claim}
%   \newtheorem*{slemma}{Lemma}
%   \newtheorem*{sprop}{Proposition}
%   \newtheorem*{notation}{Notation}

  \newtheorem{inquestion}{Question}
  \newtheorem*{sque}{Question}

%   % \newtheorem{definition}{Definition}[section]
  \newtheorem{conjecture}{Conjecture}[section]
%   \newtheorem{example}{Example}[section]
  \newtheorem*{law}{Law}

%   \newtheorem*{remark}{Remark}
%   \newtheorem*{note}{Note}
}

\newenvironment{question}[1]
{\renewcommand\theinquestion{#1}\inquestion}
{\endinquestion}

\theoremstyle{definition}{
    \newtheorem*{exercise}{Exercise}}

\sectionfont{\Large\bfseries\sffamily}
\subsectionfont{\large\bfseries\sffamily}
\subsubsectionfont{\bfseries\sffamily}
\paragraphfont{\bfseries\sffamily}
\partfont{\bfseries\sffamily}

%%%%%%%%%%%%%%%%%%%%%%%%%%%%%%%%%%%%%%%%%%%%%%%%%%%%

\title{\textbf{\sffamily\triposcourse{} Notes}}
\usepackage{libertine}
% \usepackage[libertine]{newtxmath}
\usepackage{inconsolata}

\theoremstyle{plain}{
  \newtheorem{theorem}{Theorem}[section]
  \newtheorem{lemma}[theorem]{Lemma}
  \newtheorem{proposition}[theorem]{Proposition}
  \newtheorem{corollary}[theorem]{Corollary}
  \newtheorem*{claim}{Claim}
  \newtheorem*{slemma}{Lemma}
  \newtheorem*{sprop}{Proposition}
  \newtheorem{conjecture}{Conjecture}[section]
  \newtheorem*{law}{Law}
  \newtheorem{inquestion}{Question}
  \newtheorem*{sque}{Question}
}

\theoremstyle{definition}{
  \newtheorem{method}[theorem]{Method}
  \newtheorem{technique}[theorem]{Technique}
  \newtheorem{definition}{Definition}[section]
  \newtheorem{example}{Example}[section]
  \newtheorem*{notation}{Notation}
  \newtheorem*{exercise}{Exercise}
  \newtheorem{problem}[theorem]{Problem}
  \newtheorem{implementation}[theorem]{Implementation}
}

\theoremstyle{remark}{
  \newtheorem{remark}[theorem]{Remark}
  \newtheorem*{note}{Note}
}

\newenvironment{question}[1]
{\renewcommand\theinquestion{#1}\inquestion}
{\endinquestion}

\title{\textbf{\triposcourse{} Notes}}
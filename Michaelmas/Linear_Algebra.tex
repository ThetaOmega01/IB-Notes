\documentclass[a4paper]{article}
\renewcommand{\epsilon}{\varepsilon}
\newcommand{\triposcourse}{Linear Algebra}
\input{../header.tex}
\graphicspath{ {./images/} }
\pgfplotsset{compat=1.17}
\begin{document}
\maketitle
\tableofcontents
\newpage
\part*{Lecture 1}
\section{Vector Spaces}
\subsection{Vector spaces and subspaces}
Let $ \bbF $ be an arbitrary field ($ \mathbb{R}  $ or $ \mathbb{C}  $).

\begin{definition}[$ \bbF $ Vector Space]
    An \textit{$ \bbF$ vector space} (or a \textit{vector space over $\bbF$}) is a an abelian group $ (V,+) $ with a function $ \bbF\times V \mapsto V $, defined by $ (\lambda, \bfv) \mapsto \lambda\bfv $, such that 
    \begin{enumerate}
        \item $ \lambda(\bfv_1+\bfv_2)=\lambda\bfv_1+\lambda\bfv_2 $,
        \item $ (\lambda_1+\lambda_2)\bfv = \lambda_1\bfv+\lambda_2\bfv $,
        \item $ \lambda(\mu\bfv)=(\lambda\mu)\bfv $,
        \item $ 1\bfv=\bfv $,
    \end{enumerate}
    for all $ \lambda,\mu\in \bbF$ and $ \bfv\in V $.
\end{definition}

\begin{example}
    \begin{enumerate}
        \item $ \bbF^n,\ n\in \mathbb{N}  $ is a vector space.
        \item $ \bbR^X $, where $X$ is a set, is a vector space.
        \item $ \mcM_{n,m}(\bbF) $ is a vector space.
    \end{enumerate}
\end{example}
\begin{remark}
    The axioms imply that $ \forall \bfv\in V,\ 0\cdot\bfv=0 $.
\end{remark}

\begin{definition}[Subspace]
    Let $V$ be a vector space on $\bbF$. A subset $U$ of $V$ is a \textit{vector subspace} if 
    \begin{enumerate}
        \item $ \mathbf{0}\in U $,
        \item $ (\bfu_1,\bfu_2)\in U\times U \Rightarrow u_1+u_2\in U $,
        \item $ (\lambda,\bfu)\in \bbF\times U \Rightarrow \lambda\bfu\in U $.
    \end{enumerate}
    Equivalently, $U$ is a subspace of $V$ if 
    \[
        \forall (\lambda,\mu)\in \bbF\times \bbF,\ \forall (\bfu,\bfv)\in U\times U,\ \lambda\bfu+\mu\bfv\in U.
    \]
    Denote $ U\le V $.
\end{definition}
Recall from Vectors and Matrices: the two definitions are equivalent.
\begin{proposition}
    Let $ V $ be a vector space over $\bbF$. If $ U\le V $, then $U$ is a vector space over $\bbF$.
\end{proposition}
\begin{example}
    \begin{enumerate}
        \item $ \mathbb{P}(\bbR)\le \mcC(\bbR)\le \mathbb{R} ^\mathbb{R}  $.
        \item The set of vectors
        $$
        \left\{\begin{pmatrix}x_1 \\ x_2 \\ x_3\end{pmatrix} : x_1, x_2, x_3 \in \bbR, x_1 + x_2 + x_3 = t\right\}
        $$
        is a subspace of $\bbR^3$ for $t = 0$ only.
    \end{enumerate}
\end{example}
\begin{proposition}[Intersecting Subspaces]
    Let $U, W \le V$. Then $U \cap W \le V$.
\end{proposition}
\begin{proof}
    Since $0 \in U$ and $0 \in W$, we have $0 \in U \cap W$. Now if $\lambda_1, \lambda_2 \in \bbF$ and $\bfv_1, \bfv_2 \in U \cap W$, then $\lambda_1 \bfv_1 + \lambda_2 \bfv_2 \in U$ and $V$, and thus is in $U \cap V$. Thus $U \cap W \le V$.
\end{proof}
The union of two subspaces is generally \emph{not} a subspace, as it is typically not closed by addition. In fact, the union is only ever a subspace if one of the subspaces is contained in the other. 
\begin{center}
    \includegraphics[scale=0.13]{la1.jpeg}
\end{center}

\begin{definition}[Sum of vector spaces]
    Let $V$ be a vector space over $\bbF$ and let $U,W\le V$. The \textit{sum} of $U,W$ is defined as 
    \[
        U+W  =\left\{ \bfu+\bfw: (\bfu,\bfw)\in U\times W \right\}.
    \] 
\end{definition}
For example, the sum of $x$-axis and $y$-axis is $ \mathbb{R}^{2} $.

\begin{proposition}
    $U+W\le V$.
\end{proposition}
\begin{proposition}
    $U+W$ is the smallest subspace of $V$ that contains $U,W$.
\end{proposition}
\begin{proof}
    Let $X$ be a subspace of $V$ that contains $U,W$. By closure, $ \bfu+\bfw\in X $ for all $ \bfu\in U,\bfw\in W $, so $ U+W\le X $. Hence it is the smallest subspace.
\end{proof}
\subsection{Subspaces and quotient}
\begin{definition}[Quotient]
    Let $V$ a vector space over $\bbF$, and let $U \leq V$. The \emph{quotient space} $V/U$ is the abelian group $V/U$ equipped with the scalar multiplication $\bbF \times V/U \rightarrow V/U$, $(\lambda, \bfv + U) \mapsto \lambda \bfv + U$.
\end{definition}
The multiplication is well-defined, since if $ \bfv_1+U=\bfv_2+U $ then $ \bfv_1-\bfv_2\in U $, which implies $ \lambda(\bfv_1-\bfv_2) \in U$, and thus $ \lambda\bfv_1+U=\lambda\bfv_2+U\in V/U $.
\begin{proposition}
    $V/U$ is a vector space over $\bbF$.
\end{proposition}
\begin{proof}
    Let $ \lambda_1,\lambda_2\in \bbF $ and let $ \bfv_1+U,\bfv_2+U\in V/U $. Note that 
    \begin{align*}
        \lambda_1(\bfv_1+U)+\lambda_2(\bfv_2+U) &= (\lambda_1\bfv_1+U)+(\lambda_2\bfv_2+U)\\ 
        &= \lambda_1\bfv_1+\lambda_2\bfv_2+U\in V/U.\qedhere
    \end{align*}
\end{proof}
\end{document}
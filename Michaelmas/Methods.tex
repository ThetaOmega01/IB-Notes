\documentclass[a4paper]{article}
\renewcommand{\epsilon}{\varepsilon}
\newcommand{\triposcourse}{Methods}
\input{../header.tex}
\graphicspath{ {./images/} }
\pgfplotsset{compat=1.17}
\begin{document}
\maketitle
\tableofcontents
\newpage
\part*{Lecture 1}
\part{Self-adjoint ODEs}
\section{Fourier Series}
\subsection{Periodic functions}
\begin{definition}
    A function $f$ is \textit{periodic} if $ f(x+T)=f(x), \forall x $, where $T$ is the period.
\end{definition}
\begin{center}
    \includegraphics[scale=0.1]{methods1.jpeg}
\end{center}
Consider the set of functions 
\[
    g_n(x) = \cos \left( \frac{n\pi x}{L} \right),\quad h_n(x) = \sin \left( \frac{n\pi x}{L} \right).
\]
They are periodic on $ [0,2L] $ with period 2L.

Recall identities
\[
    \begin{aligned} \cos (A \pm B) &=\cos A \cos B \mp \sin A \sin B \\ \sin (A \pm B) &=\sin A \cos B \pm \cos A \sin B, \text { and so } \\ \cos A \cos B &=\frac{1}{2}[\cos (A-B)+\cos (A+B)] \\ \sin A \sin B &=\frac{1}{2}[\cos (A-B)-\cos (A+B)] \\ \sin A \cos B &=\frac{1}{2}[\sin (A+B)+\sin (A-B)] \end{aligned}
\]
Define an inner product of the set 
\[
    \langle f,g \rangle = \int_{0}^{2L} f(x)g(x)\,\mathrm{d}x.
\]
\begin{proposition}
    With this inner product, $g_n,h_n$ are mutually orthogonal on $ [0,2L) $.
\end{proposition}
\begin{proof}
    Check that
    \[
        \langle h_n,h_m \rangle = \begin{cases}
            L\delta_{nm} &\forall n,m\neq 0\\
            0 & m=0 \lor n=0\\
            \end{cases} \quad
            \langle g_n,g_m \rangle = \begin{cases}
            L \delta_{mn} &n,m\neq 0\\
            2L \delta_{0n} & m=0\\
            \end{cases} \quad
            \langle h_n,g_m \rangle =0.
    \]
\end{proof}
$g_n,h_n$ form a complete orthogonal set, i.e. they span the space of well-behaved periodic functions $ [0,2L) $ and they are linearly independent.

\subsection{Definition of Fourier series}
We can express any well-behaved periodic function of period 2L as 
\[
    f(x) = \frac{1}{2}a_0 + \sum_{n=1}^{\infty} a_n \cos \left( \frac{n\pi x}{L} \right)+ \sum_{n=1}^{\infty} b_n \sin \left( \frac{n\pi x}{L} \right)\tag{$*$}
\]
where $a_n,b_n$ are constants such that RHS is convergent for all $x$ where $f$ is continuous. At a discontinuity $x$, the Fourier series approaches the midpoint of upper and lower limits at $x$, i.e. 
\[
    \frac{f(x_+)+f(x_-)}{2} = \frac{1}{2}a_0 + \sum_{n=1}^{\infty} a_n \cos \left( \frac{n\pi x}{L} \right)+ \sum_{n=1}^{\infty} b_n \sin \left( \frac{n\pi x}{L} \right).
\]
Consider $ \langle h_n,f \rangle  $ and substitute in ($ * $), we get 
\[
    \int_{0}^{2L} \sin \left( \frac{n \pi x}{L} \right) f(x) \,\mathrm{d}x = \sum_{m=1}^{\infty}Lb_m \delta_{nm} = Lb_n.
\]
Similar for $a_n$. We get
\[
    b_{n}=\frac{1}{L} \int_{0}^{2 L} f(x) \sin \left(\frac{n \pi x}{L}\right)\, \rmd x,\quad a_n = \frac{1}{L}\int_{0}^{2L} f(x) \cos \left( \frac{n \pi x}{L} \right) \,\mathrm{d}x.
\]
\begin{note}
    \begin{enumerate}
        \item $a_n$ includes $n=0$ since $ \frac{1}{2}a_0 $ is the average of $f$, i.e.
        \[
            \langle f \rangle = \frac{1}{2L}\int_{0}^{2L} f(x) \,\mathrm{d}x.
        \]
        \item Range of integration can be changed as long as it has length $2L$. We usually consider $ \int_{-L}^{L} $.
        \item Can think of the Fourier series as the decomposition into harmonics. The simplest Fourier series are sines and cosines. e.g. $ \sin (3 \pi x/ L) $ has Fourier series $b_3=1$ and $ b_n=0,n\neq 3 ,a_n=0$.
    \end{enumerate}
\end{note}
\begin{example}[Sawtooth wave]
    Consider $f(x)=x,x\in [-L,L)$ and periodic elsewhere.
    \begin{center}
        \includegraphics[scale=0.1]{methods2.jpeg}
    \end{center}
    We have 
    \begin{align*}
        a_n &= \frac{1}{L}\int_{-L}^{L} x\cdot \cos \left( \frac{n\pi x}{L} \right) \,\mathrm{d}x=0,\\ 
        b_n &= \frac{2}{L} \int_{0}^{L} x\cdot \sin \left( \frac{n\pi x}{L} \right) \,\mathrm{d}x = \frac{2L}{n\pi }(-1)^{n+1}.
    \end{align*}
    So the Fourier series is 
    \[
        f(x) = \frac{2L}{\pi } \sum_{n=1}^{\infty}\frac{(-1)^{n+1}}{n}\sin \left( \frac{n\pi x}{L} \right).
    \]
\end{example}
\begin{note}
    As $n\to \infty$ in Fourier series,
    \begin{enumerate}
        \item The FS approximation improves (convergent where continuous).
        \item FS $ \to 0 $ at $x=L$, i.e. the midpoint of continuity.
        \item FS has a persistent \textit{overshoot} at $x=L$, which is approximately 9\% and is known as the \textit{Gibbs phenonenon}. See Example Sheet Q5.
    \end{enumerate}
\end{note}
\newpage
\part*{Lecture 2}
\subsection{The Dirichlet Conditions and Fourier's Theorem}
A natural question is then which functions are allowed to have a proper Fourier series.
Surprisingly, a big, yet hard to precisely characterise, class of functions has a convergent Fourier series that has the desired properties.
This class even includes some classical counterexamples in analysis.
As an applied course, we will just look at some of the sufficient conditions.
\begin{theorem}[Fourier's Theorem]
    If $f$ is a bounded periodic function with period $2L$ with a finite number of minima, maxima, and discontinuities in $[0,2L)$, then its Fourier series converges to $f$ where it is continuous and converges to the average of the two side limits.
\end{theorem}
The conditions in this theorem is known as the Dirichlet conditions.
\begin{note}
    \begin{enumerate}
        \item These conditions are hella weak compared to our conditions for a function to have e.g. a Taylor series.
        However, pathological functions like $1/x,\sin(1/x),1_{\mathbb R\setminus\mathbb Q}(x)$ are excluded from these conditions.
        \item The converse is not true, as $\sin(1/x)$ has a Fourier series we desire.
    \end{enumerate}
\end{note}
\begin{proof}
    See Jeffery \& Jeffery book.
\end{proof}
Another subject of interest is the rate of convergence of a Fourier series.
Perhaps unsurprisingly, it depends on the smoothness of the function.
\begin{theorem}
    If $f(x)$ is $p^{th}$ differentiable but $f^{(p)}$ is not continuous, then its Fourier series converges as $O(n^{-(p+1)})$ as $n\to\infty$.
\end{theorem}
\begin{example}
    \begin{enumerate}
        \item  Consider the square wave
        $$f(x)=\begin{cases}
            1\text{, for $0\le x<1$}\\
            -1\text{, for $-1\le x<0$}
        \end{cases}$$
        That extends periodically with period $2$.
        Then it has a Fourier series
        $$4\sum_{m=1}^\infty\frac{\sin[(2m-1)\pi x]}{(2m-1)\pi}$$
        which, as one can see both from the preceding theorem (with $p=0$) and observation, converges slowly.
        \item Consider the general ``see-saw'' wave
        $$f(x)=\begin{cases}
            x(1-\xi)\text{, for $x\in[0,\xi)$}\\
            \xi(1-x)\text{, for $x\in[\xi,1)$}
        \end{cases}$$
        which extends as an odd periodic function with period $2$.
        This has Fourier series
        $$2\sum_{m=1}^\infty\frac{\sin (n\pi\xi)\sin (n\pi x)}{(n\pi)^2}$$
        which converges with $p=1$ in the preceding theorem.
        In particular, $\xi=1/2$ gives
        $$2\sum_{m=1}^\infty(-1)^{m+1}\frac{\sin[(2m-1)\pi x]}{[(2m-1)\pi]^2}$$
        which can be seen, immediately, that it converges faster than the series in the previous example.
        \item Take $f(x)=x(1-x)/2$ for $x\in[0,1)$ that extends as an odd periodic function with period $2$.
        Then its Fourier series is
        $$4\sum_{m=1}^\infty\frac{\sin[(2m-1)\pi x]}{[(2m-1)\pi]^3}$$
        which has $p=2$.
        \item Take $f(x)=(1-x^2)^2$, then $a_n=O(n^{-4})$.
    \end{enumerate}
\end{example}
Of course, we want to integrate and differentiate a Fourier series term-by-term.
Integration, as one expect, seldom yields problems as it imposed very few restrictions on the function.

And indeed, we are just going to assume we can integrate any Fourier series term-by-term and they guarantee to yield a smoother function, which satisfies the Dirichlet conditions if the original function does.

Differentiation is more problematic when doing it term-by-term.
\begin{example}
    Take the square wave again which is known to have Fourier series
    $$4\sum_{m=1}^\infty\frac{\sin[(2m-1)\pi x]}{(2m-1)\pi}$$
    which, after term-by-term differentiation, yields
    $$4\sum_{m=1}^\infty\cos[(2m-1)\pi x]$$
    which is clearly divergent.
    This is perhaps unsurprising as the original function is not even continuous.
\end{example}
\begin{theorem}
    If $f(x)$ is differentiable and both $f,f^\prime$ satisfy Dirichlet conditions, then we can differentiate the Fourier series of $f$ term-by-term to get the Fourier series of $f^\prime$.
\end{theorem}
\begin{example}
    If we differentiate the see-saw curve with $\xi=1/2$, then we will get an offset of the Fourier series of the square wave with $ \tilde{x}=x+1/2 $.
\end{example}
\subsection{Parseval's Theorem}
There is some interesting relation between the integral of the square of a function and the square of the Fourier coefficients of that function.
If the function is nice enough to have a nice enough Fourier series, then by orthogonality,
\begin{align*}
    \int_0^{2L}f(x)^2\,\mathrm dx&=\int_0^{2L}\left( \frac{a_0}{2}+\sum_{n\ge 1}a_n\cos\frac{n\pi x}{L}+\sum_{n\ge 1}b_n\sin\frac{n\pi x}{L} \right)^2\,\mathrm dx\\
    &=\int_0^{2L}\left( \frac{a_0^2}{4}+\sum_{n\ge 1}a_n^2\cos^2\frac{n\pi x}{L}+\sum_{n\ge 1}b_n^2\sin^2\frac{n\pi x}{L} \right)\,\mathrm dx\\
    &=L\left( \frac{a_0^2}{2}+\sum_{n\ge 1}(a_n^2+b_n^2) \right).
\end{align*}
This is also called the \textit{completeness relation} as the left hand side would be greater than or equal to the right hand side if any basis functions are missing from the series.
This is known as Parseval's Theorem.
\begin{theorem}[Parseval's Theorem]\label{parseval}
    For a nice enough function $f$ with Fourier coefficients $a_n,b_n$, we have
    $$\int_0^{2L}f(x)^2\,\mathrm dx=L\left( \frac{a_0^2}{2}+\sum_{n\ge 1}(a_n^2+b_n^2) \right)$$
\end{theorem}
\begin{proof}
    Above.
\end{proof}
\begin{example}
    Consider the sawtooth curve with $f(x)=x,x\in[-L,L)$ with period $2L$.
    Then Parseval's Theorem reveals that
    $$\frac{2}{3}L^3=\int_{-L}^Lx^2\,\mathrm dx=L\sum_{n=1}^\infty\frac{4L^2}{n^2\pi^2}=\frac{4L^3}{\pi^2}\sum_{n=1}^\infty\frac{1}{n^2}\implies \sum_{n=1}^\infty\frac{1}{n^2}=\frac{\pi^2}{6}.$$
\end{example}
\begin{remark}
    If we think of the integral of the square as $ \langle f,f \rangle = \left\| f \right\| ^2  $, the $ L^2 $ norm, then Parseval's Theorem can be thought of an analog of Pythagoras' Theorem in this space of functions.
\end{remark}
\newpage
\part*{Lecture 3}
\subsection{Alternative Fourier Series}
Consider a function $f:[0,L)\to\mathbb R$.
We can extend $f$ to a periodic function of period $2L$ in two ways:
\begin{enumerate}
    \item We can require the function to be odd, then $a_n=0$ for all $n$ and
    $$b_n=\frac{2}{L}\int_0^Lf(x)\sin\frac{n\pi x}{L}\,\mathrm dx$$
    and the Fourier series would be $\sum_{n\ge 1}b_n\sin(n\pi x/L)$, which is called a \textit{Fourier sine series}.
    The sawtooth function is an example of this.
    \item We can require the function to be even, then $b_n=0$ for all $n$ and
    $$a_n=\frac{2}{L}\int_0^Lf(x)\cos\frac{n\pi x}{L}\,\mathrm dx$$
    So the Fourier series is $a_0/2+\sum_{n\ge 1}a_n\cos(n\pi x/L)$.
    This is called a \textit{Fourier cosine series}.
    $f(x)=(1-x^2)^2$ is an example (where $L=1$).
\end{enumerate}
The actual thing we want is to represent the Fourier series more neatly in terms of exponentials.
We know that
$$\cos\frac{n\pi x}{L}=\frac{e^{in\pi x/L}+e^{-in\pi x/L}}{2},\sin\frac{n\pi x}{L}=\frac{e^{in\pi x/L}-e^{-in\pi x/L}}{2i}$$
So by writing $c_0=a_0/2$ and
$$c_m=\begin{cases}
    (a_m-ib_m)/2&\text{for $m>0$}\\
    (a_{-m}+ib_{-m})/2&\text{for $m<0$}
\end{cases}$$
we obtain
$$\frac{a_0}{2}+\sum_{n=1}^\infty a_n\cos\frac{n\pi x}{L}+\sum_{n=1}^\infty b_n\sin\frac{n\pi x}{L}=\sum_{m=-\infty}^\infty c_me^{im\pi x/L}$$
Equivalently, if we extend our inner product to the complex functions
$$\langle f,g\rangle=\int_{-L}^Lf(x)\overline{g(x)}\,\mathrm dx$$
Then $\langle e^{im\pi x/L},e^{in\pi x/L}\rangle=2L\delta_{mn}$, which means they are orthogonal as well and we can then obtain
$$c_m=\frac{1}{2L}\langle f(x),e^{im\pi x/L}\rangle=\frac{1}{2L}\int_{-L}^Lf(x)e^{-im\pi x/L}\,\mathrm dx$$
By thinking them as a set of basis of a space of nice-enough functions in the way we did for $\sin$ and $\cos$.
Parseval's Theorem can then be stated as
$$\int_{-L}^L|f(x)|^2\,\mathrm dx=2L\sum_{n=-\infty}^\infty|c_n|^2$$
\subsection{Some Motivations of Fourier Series}
\begin{definition}
    The complex inner product $\langle\cdot\rangle:\mathbb C^N\times\mathbb C^N\to\mathbb C$ is defined by
    $$\langle\mathbf{u},\mathbf{v}\rangle=\mathbf{u}^\dagger\mathbf{v}$$
\end{definition}
An $N\times N$  matrix $A$ is self-adjoint (or Hermitian) if
$$\forall\mathbf{u},\mathbf{v}\in\mathbb C^N,\langle A\mathbf{u},\mathbf{v}\rangle=\langle\mathbf{u},A\mathbf{v}\rangle.$$
One can show that this is just saying $A^\dagger=A$.
It can be shown that $A$ satisfies:
\begin{enumerate}
    \item All eigenvalues are real for all $n$.
    \item Eigenvectors associated with different eigenvalues are orthogonal with respect to $\langle\cdot\rangle$.
    \item We can rescale to make an orthonormal basis of $\mathbb C^N$ of eigenvectors $\{\mathbf{v}_1,\ldots,\mathbf{v}_N\}$.
\end{enumerate}

Now, given any $\mathbf{b}$, if we want to solve for $\mathbf{x}$ in $A\mathbf{x}=\mathbf{b}$, then a way to do it is to express $\mathbf{b}=\sum_nb_n\mathbf{v}_n$ and observe that if $\sum_nc_n\mathbf{v}_n$ is a solution then
$$\sum_nb_n\mathbf{v}_n=A\left(\sum_{n=1}^Nc_n\mathbf{v}_n\right)=\sum_{n=1}^Nc_n\lambda_n\mathbf{v}_n$$
where $\lambda_n$ is the eigenvalue associated with $\mathbf{v}_n$.
So if $A$ is nonsingular, then none of the $\lambda_n$ is zero and we can write $c_n=b_n/\lambda_n$ and get the solution
$$\mathbf{x}=\sum_{n=1}^N\frac{b_n}{\lambda_n}\mathbf{v}_n$$
This means we can solve an linear equation if there is a basis consisting of eigenvectors of the matrix.
We want an analogy of this in solving linear ODEs.
Consider the differential operator
$$\mathcal Ly=-\frac{\mathrm d^2y}{\mathrm dx^2}$$
and suppose we want to solve $\mathcal Ly=f(x)$ for a function $f(x)$ subject to boundary conditions $y(0)=y(L)=0$.
The related eigenvalue problem is then $\mathcal L y_n=\lambda_ny_n$ with $y_n(0)=y_n(L)=0$ which has solutions
$$y_n(x)=\sin\frac{n\pi x}{L},\lambda_n=\left( \frac{n\pi}{L} \right)^2$$
So we will want to write
$$y(x)=\sum_{n=1}^\infty c_n\sin\frac{n\pi x}{L},f(x)=\sum_{n=1}^\infty b_n\sin\frac{n\pi x}{L},b_n=\frac{2}{L}\int_0^Lf(x)\sin\frac{n\pi x}{L}\,\mathrm dx$$
and ignore every convergence problem.
Then this substitution yields
$$\sum_{n=1}^\infty b_n\sin\frac{n\pi x}{L}=\mathcal Ly=-\frac{\mathrm d^2y}{\mathrm dx^2}\left( \sum_{n=1}^\infty c_n\sin\frac{n\pi x}{L} \right)=\sum_{n=1}^\infty c_n\left( \frac{n\pi}{L} \right)^2\sin\frac{n\pi x}{L}$$
Hence, $c_n=b_n(L/(n\pi))^2$ by orthogonality, so we can get a particular solution of the problem in the form
$$y(x)=\sum_{n=1}^\infty \frac{b_n}{\lambda_n}y_n$$
which is the analogy we wanted.
\begin{example}\label{odd_sq_fourier_ode}
    Let $L=1$ and set $f$ to be the odd square wave with $f(x)=1$ for $x\in[0,1)$.
    This has Fourier series
    $$4\sum_{m=1}^\infty\frac{\sin[(2m-1)\pi x]}{(2m-1)\pi}$$
    So the above discussion instantly yield a solution
    $$y(x)=\sum_{n=1}^\infty \frac{b_n}{\lambda_n}y_n=4\sum_{m=1}^\infty\frac{\sin[(2m-1)\pi x]}{[(2m-1)\pi]^3}$$
    which is the Fourier series of $y(x)=x(1-x)/2$ on $[0,1)$ extending as an odd periodic function with period $2$.

    Indeed, as one can verifty, if we integrate $\mathcal L y=1$ directly with the appropriate boundary conditions, we can get basically the same solution.
\end{example}
\subsection{A Glimpse into Green's Functions}
Fix $L=1$ and consider an odd function $f$.
We have
\begin{align*}
    y(x)&=\sum_{n=1}^\infty\frac{b_n}{\lambda_n}\sin(\pi x)\\
    &=\sum_{n=1}^\infty\frac{2}{(n\pi)^2}\left(\int_0^1f(\xi)\sin(n\pi\xi)\,\mathrm d\xi\right)\sin(n\pi x)\\
    &=\int_0^12\sum_{n=1}^\infty\frac{\sin(n\pi x)\sin(n\pi\xi)}{(n\pi)^2}f(\xi)\,\mathrm d\xi\\
    &=\int_0^1G(x,\xi)f(\xi)\,\mathrm d\xi
\end{align*}
where
$$G(x,\xi)=2\sum_{n=1}^\infty\frac{\sin(n\pi x)\sin(n\pi\xi)}{(n\pi)^2}$$
It is exactly the general see-saw wave
$$G(x,\xi)=\begin{cases}
    x(1-\xi)\text{, for $x\in[0,\xi)$}\\
    \xi(1-x)\text{, for $x\in[\xi,1)$}
\end{cases}$$
This is the Green's function for this ODE $-\rmd^2 y/\rmd x^2 =f(x)$.
One can actually solve this integral and get what we got in Example \ref{odd_sq_fourier_ode}.
\newpage
\part*{Lecture 4}
\section{Sturm-Liouville Theory}
\subsection{Review of Second-Order Linear ODEs}
For a general inhomogeneous ODE $\mathcal Ly=f(x)$ where
$$\mathcal Ly=\alpha(x)\frac{\mathrm d^2y}{\mathrm dx^2}+\beta(x)\frac{\mathrm dy}{\mathrm dx}+\gamma(x)y$$
In general, the homogeneous equation $\mathcal Ly=0$ has two linearly independent solutions $y_1,y_2$.
The complementary function $y_c(x)=Ay_1+By_2$ for constants $A,B$ is then the general solution to $\mathcal Ly=0$ by linearity.

If we can find a particular solution (aka particular integral) $y_p$ to $\mathcal Ly=f$, then $y_p+y_c=y_p+Ay_1+By_2$ for $A,B$ constants is the general solution to $\mathcal Ly=f$ again by linearity.
Two pieces of boundary data is then needed to determine the constants $A,B$.

There are several types of boundary conditions.
We sometimes get the Dirichlet condition of specifying the function's value at the endpoints, or the Neumann condition of specifying the derivative's values at the endpoints.
Sometimes these two types of conditions are mixed.

The sort of conditions we often consider are homogeneous conditions, i.e. the function vanishes at the endpoints.
The reason of it is that it allows the superposition of solutions in a linear DE.
What if we come across a inhomogeneous condition?
We can use the complementary solution to cancel stuff out.

Sometimes, we specify initial data of the function and its derivative as boundary conditions.

Another matter of interest is the general eigenvalue problem.
To solve $\mathcal Ly=f$ using eigenvalue decompositions like we did previously, we must first solve (subject to boundary conditions) the related eigenvalue problem
$$\mathcal Ly=\alpha(x)\frac{\mathrm d^2y}{\mathrm dx^2}+\beta(x)\frac{\mathrm dy}{\mathrm dx}+\gamma(x)y=-\lambda\rho(x)y$$
where $\rho$ is nonegative.
This form often occurs after seperation of variables in a PDE.
\subsection{Self-Adjoint Operators}
\begin{definition}
    For two functions $f,g:[a,b]\to\mathbb C$ we define their inner product to be
    $$\langle f,g\rangle=\int_a^bf^*(x)g(x)\,\mathrm dx$$
\end{definition}
We can guarantee to rewrite the original eigenvalue problem into the Sturm-Liouville form, i.e. $\mathcal Ly=\lambda wy$ where we are able to rewrite $\mathcal Ly=-(py^\prime)^\prime+qy$ and $w$ is a nonnegative weight function.
\footnote{The reason why there is a weight function there is just for convenience.}
How to convert a second order linear ODE to this form?
Simply multiply the diffential equation by an integrating factor $F$ that will be specified later and we can write
$$\frac{\mathrm d}{\mathrm dx}(F\alpha y^\prime)-F^\prime\alpha y^\prime-F\alpha^\prime y^\prime+F\beta y^\prime+F\gamma y=-\lambda F\rho y$$
So to eiminate the $y^\prime$ terms, we set
$$F(x)=\exp\left(\int\frac{\beta-\alpha^\prime}{\alpha}\,\mathrm dx\right)$$
which reduced the equation to
$$(F\alpha y^\prime)^\prime+F\gamma y=-\lambda F\rho y$$
Setting $p=F\alpha,q=F\gamma$ and $w=F\rho\ge 0$.
\begin{example}
    Consider the Hermite equation that appears in quantum mechanics
    $$y^{\prime\prime}-2xy^\prime+2ny=0$$
    Then $\alpha=1,\beta=-2x,\gamma=0,\lambda\rho=2n$, so the above procedure translates this to the Sturm-Liouville form
    $$\mathcal L=(-e^{-x^2}y^\prime)^\prime=2ne^{-x^2}y$$
\end{example}
\begin{definition}
    Let $\mathcal L:C\to C$ be an operator, where $C$ on a class of functions $[a,b]\to\mathbb C$ equipped with the inner product we defined previously.
    This operator $\mathcal L$ is self-adjoint if $\langle y_1,\mathcal Ly_2\rangle=\langle\mathcal Ly_1,y_2\rangle$ for any $y_1,y_2\in C$.
\end{definition}
If we let $\mathcal L$ be the operator in the Strum-Liouville form, then
\begin{align*}
    \langle y_1,\mathcal Ly_2\rangle-\langle\mathcal Ly_1,y_2\rangle&=\int_a^b[-y_1(py_2^\prime)^\prime+y_1qy_2+y_2(py_1^\prime)^\prime-y_2qy_1]\,\mathrm dx\\
    &=\int_a^b[-y_1(py_2^\prime)^\prime+y_2(py_1^\prime)^\prime]\,\mathrm dx\\
    &=\int_a^b[-(y_1(py_2^\prime)^\prime+y_1^\prime py_2^\prime)+(y_2(py_1^\prime)^\prime+y_2^\prime py_1^\prime)]\,\mathrm dx\\
    &=\int_a^b[-(py_1y_2^\prime)^\prime+(py_1^\prime y_2)^\prime]\,\mathrm dx\\
    &=[-py_1y_2^\prime+py_1^\prime y_2]_a^b
\end{align*}
So for this operator to be self-adjoint, we need some good enough boundary conditions so that enough stuff vanishes.
This includes homogeneous boundary condition $y(a)=y(b)=0$ or $y^\prime(a)=y^\prime(b)=0$ or mixed $y+ky^\prime=0$ etc..
We say a Sturm-Liouville problem is regular if the boundary conditions are homogeneous.
Periodic boundary conditions also work, where we can take $y(a)=y(b)$ and the derivatives are specified (or periodic) at the boundary.
There can also be singular points of this ODE, where $p(a)=p(b)=0$.
We can have combinations of above too.

\subsection{Properties of Self-Adjoint Operators}
\begin{definition}
    The inner product of $y_1,y_2:[a,b]\to\mathbb C$ with respect to weight $w:[a,b]\to\mathbb R_{\ge 0}$ is
    $$\langle f,g\rangle_w=\int_a^bwf^*g\,\mathrm dx=\langle wf,g\rangle=\langle f,wg\rangle$$
\end{definition}
Analogous to the finite dimensional case, we have
\begin{theorem}\label{self-adjoint}
    For a sufficiently nice self-adjoint operator $\mathcal L$ on a sufficiently nice space of functions:\\
    (a) Eigenvalues of $\mathcal L$ are real.\\
    (b) Eigenfunctions of it with different eigenvalues are orthogonal with respect to the weight $w$.\\
    (c) We can take the eigenfunctions as a set of basis for the function space, just like Fourier series.
\end{theorem}
\begin{proof}[Proof of (a)]
    If $\mathcal Ly=\lambda wy$, taking complex conjugate gives $\mathcal Ly^*=\lambda^*wy^*$.
    Hence as $\mathcal L$ is self-adjoint,
    $$0=\int_a^b(y^*\mathcal Ly-y\mathcal Ly^*)\,\mathrm dx=(\lambda-\lambda^*)\int_a^bw|y|^2\,\mathrm dx$$
    which means $\lambda=\lambda^*$, so $\lambda$ is real.
\end{proof}
If $\lambda$ is non-degenerate (simple), i.e. it has a one-dimensional eigenspace, then $y$ is guaranteed to be real.
Even if it has dimension $2$ (not more because the ODE is second order), we can still find two real functions as basis of the eigenspace.
Also, by considering $u\mathcal Lv-v\mathcal Lu=(-p(uv^\prime-u^\prime v))^\prime$, one can show that a regular Sturm-Liouville problem always has all eigenvalues simple.
\begin{proof}[Proof of (b)]
    Suppose $\mathcal Ly_m=\lambda_mwy_m$ and $\mathcal Ly_n=\lambda_nwy_n$, then
    $$0=\int_a^by_n\mathcal Ly_m-y_m\mathcal Ly_n\,\mathrm dx=(\lambda_m-\lambda_n)\int_a^bwy_ny_m\,\mathrm dx$$
    But $\lambda_m$ and $\lambda_n$ are distinct.
    The claim follows.
\end{proof}
As an aside, we do not really need the weight function in order to formulate Sturm-Liouville theory, since we can do the transformation $\tilde{y}=\sqrt{w}y$ and replace $\mathcal Ly$ by $(1/\sqrt{w})\mathcal L(\tilde{y}/\sqrt{w})$.
Yet the analytic property is generally simpler if we keep $w$.
\end{document}
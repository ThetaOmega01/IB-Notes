\documentclass[a4paper]{article}
\renewcommand{\epsilon}{\varepsilon}
\newcommand{\triposcourse}{Analysis and Topology}
\input{../header.tex}
\graphicspath{ {./images/} }
\pgfplotsset{compat=1.17}
\begin{document}
\maketitle
\tableofcontents

\part{Generalizing continuity and convergence}
\section{Three examples of convergence}
\subsection{Convergence in $\mathbb{R}$}
Recall from IA: 
\begin{definition}
    Let $(x_n)$ be a sequence in $\mathbb{R}$ and $x\in \mathbb{R}$. We say $(x_n)$ converges to $x$ and write $x_n\to x$ if 
    \[
        \forall \epsilon>0\quad \exists N\quad \forall n\ge N\quad |x_n-x|<\epsilon.
    \]
\end{definition}

Useful fact: $ \forall a,b\in \mathbb{R}, |a+b|\le |a|+|b| $. 

Recall Bolzano-Weierstrass theorem:
\begin{theorem}[BWT]
    A bounded sequence in $\mathbb{R}$ must have a convergent subsequence. 
\end{theorem}

Recall that 
\begin{definition}
    A sequence $(x_n)$ is Cauchy if 
    \[
        \forall \epsilon>0\quad \exists N\quad \forall m,n\ge N\quad |x_m-x_n| < \epsilon. 
    \]
\end{definition}
and GPC: 
\begin{theorem}
    Any Cauchy sequence in $\mathbb{R}$ converges. 
\end{theorem}
See IA Analysis for proofs. 

\subsection{Convergence in $ \mathbb{R}^{2} $}
\begin{remark}
    This all works in $ \mathbb{R}^{n} $. 
\end{remark}
Let $(z_n)$ be a sequence in $ \mathbb{R}^{2} $ and $z\in \mathbb{R}^{2}$. What should $z_n\to z$ mean? 

In $\mathbb{R}$: `As $n$ gets large, $z_n$ gets arbitrarily close to $z$'
What does `close' mean in $ \mathbb{R}^{2}$?

In $\mathbb{R}:$ $a,b$ close if $|a-b|$ small. In $ \mathbb{R}^{2} $: replace $ |\cdot | $ by $ \left\| \cdot  \right\| $. 

Recall: If $\mathbf{z}=(x,y)$ then $ \left\| \mathbf{z} \right\| = \sqrt{x^2+y^2} $ and triangle inequality
\[
    \left\| \mathbf{a}+\mathbf{b} \right\| \le \left\| \mathbf{a} \right\| + \left\| \mathbf{b} \right\|
\]

\begin{definition}
    Let $\mathbf{z}_n\in \mathbb{R}^{2}$ and $\mathbf{z}\in \mathbb{R}^{2}$. We say $ \mathbf{z}_n $ converges to $\mathbf{z}$ and write $\mathbf{z}_n\to \mathbf{z}$ if 
    \[
        \forall \epsilon>0\quad \exists N\quad \forall n\ge N\quad \left\| \mathbf{z}_n-\mathbf{z} \right\| < \epsilon. 
    \]
\end{definition}
\begin{note}
    Equivalently, $ \mathbf{z}\to \mathbf{z} $ if and only if $ \left\| \mathbf{z}_n-\mathbf{z} \right\|\to 0 $. 
\end{note}

\begin{example}
    Let $ \mathbf{z}_n,\mathbf{w}_n $ be sequences in $ \mathbb{R}^{2} $ with $ \mathbf{z}_n\to \mathbf{z},\mathbf{w}_n\to \mathbf{w} $. Then $ \mathbf{z}_n+\mathbf{w}_n\to \mathbf{z}+\mathbf{w} $. 

    \begin{proof}
        Note that 
        \[
            \left\| (\mathbf{z}_n+\mathbf{w}_n)-(\mathbf{z}+\mathbf{w}) \right\|\le \left\| \mathbf{z}_n-\mathbf{z} \right\| + \left\| \mathbf{w}_n-\mathbf{w} \right\|\to 0+0=0. \qedhere
        \]
    \end{proof}
\end{example}
In fact, given convergence in $\mathbb{R}$, convergence in $ \mathbb{R}^{2} $ is easy: 
\begin{proposition}
    Let $ \mathbf{z}_n $ be a sequence in $ \mathbb{R}^{2} $ and $\mathbf{z}\in \mathbb{R}^{2}$. Write $ \mathbf{z}_n = (x_n,y_n) $ and $\mathbf{z} = (x,y)$. Then $ \mathbf{z}_n\to \mathbf{z} $ if and only if $ x_n\to x,y_n\to y $ 
\end{proposition}
\begin{proof}
    $ (\Rightarrow) $ Note that $ |x_n-x|,|y_n-y|\le \left\| \mathbf{z}_n-\mathbf{z} \right\| $. So if $ \left\| \mathbf{z}_n-\mathbf{z} \right\|\to 0 $ then $ |x_n-x|,|y_n-y|\to 0 $. 

    $ ( \Leftarrow) $ If $ |x_n-x|,|y_n-y|\to 0 $ then 
    \[
        \left\| \mathbf{z}_n-\mathbf{z} \right\| = \sqrt{(x_n-x)^2 + (y_n-y)^2}\to 0.\qedhere
    \]
\end{proof}

\begin{definition}
    A sequence $\mathbf{z}_n$ in $ \mathbb{R}^{2} $ is \textbf{bounded} if $\exists M\in \mathbb{R}$ such that $ \forall n,\ \left\| \mathbf{z}_n \right\|\le M $. 
\end{definition}

\begin{theorem}[Bolzano-Weierstrass]
    A bounded sequence in $ \mathbb{R}^{2} $ must have a convergent subsequence.
\end{theorem}
Use similar interval bisection in $ \mathbb{R}^{2} $ can do, but we will convert to $\mathbb{R}$ case. 

\begin{proof}
    Let $ \mathbf{z}_n $ be a bounded sequence in $ \mathbb{R}^{2} $. Write $ \mathbf{z}_n = (x_n,y_n) $. Now $ \forall n, |x_n| \le \left\| \mathbf{z}_n \right\| $ so $x_n$ is a bounded sequence in $ \mathbb{R} $. So by Bolzano-Weierstrass in $ \mathbb{R} $, it has a convergence subsequence, say $ x_{n_j}\to x\in \mathbb{R} $. 

    Similarly $ y_{n_j} $ is a bounded sequence in $ \mathbb{R} $ so has a convergent subsequence $ y_{n_{j_k}}\to y $. Now also $ x_{n_{j_k}} \to x$, so $ \mathbf{z}_{n_{j_k}}\to \mathbf{z}= (x,y) $. 
\end{proof}

\begin{definition}
    A sequence $ \mathbf{z}_n\in \mathbb{R}^{2} $ is \textbf{Cauchy} if 
    \[
        \forall \epsilon>0\quad \exists N\quad \forall m,n\ge N\quad \left\| \mathbf{z}_m-\mathbf{z}_n \right\|<\epsilon
    \]
\end{definition}
Easy exercise: convergent then Cauchy in $ \mathbb{R}^{2} $. 

\begin{theorem}[GPC for $ \mathbb{R}^{2} $]
    Any Cauchy sequence in $ \mathbb{R}^{2} $ converges. 
\end{theorem}
Can do similar proof as $\mathbb{R}$, but convert to $\mathbb{R}$ case. 
\begin{proof}
    Let $ \mathbf{z}_n $ be a Cauchy sequence in $ \mathbb{R}^{2} $. Write $ \mathbf{z}_n = (x_n,y_n) $. For all $ m,n, |x_m-x_n| \le \left\| \mathbf{z}_m-\mathbf{z}_n \right\| $, so $ x_n $ is Cauchy and converges by GPC. Similarly $ y_n $ converges and by proposition 1.3, $ \mathbf{z}_n $ converges.
\end{proof}
Thought of the day: what about continuity? 

Let $ f: \mathbb{R}^{2}\to \mathbb{R} $. What does it mean for $f$ to be continuous? (simple modification for 1d case)

What can we do with it? Big theorem in IA: if $f$ is a continuous function on a closed bounded interval then $f$ is bounded and attains its bounds. 

Is there a similar theorem for $ \mathbb{R}^{2}\to \mathbb{R}^{2} $? 
What do we replace for `closed and bounded interval' by? 

In $ \mathbb{R} $ we prove this theorem by BWT. Why did it work? Why did we need a closed bounded interval to make it work? What can we do in $ \mathbb{R}^{2} $? 

\subsection{Convergence of functions}

Let $X \subseteq \mathbb{R}$, and let $ f_n:X\to \mathbb{R}\ (n\ge 1) $ and let $f:X\to \mathbb{R}$. What does it mean for $ (f_n) $ to converge to $f$?

\end{document}
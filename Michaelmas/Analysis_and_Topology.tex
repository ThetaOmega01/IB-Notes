\documentclass[a4paper]{article}
\renewcommand{\epsilon}{\varepsilon}
\newcommand{\triposcourse}{Analysis and Topology}
\input{../header.tex}
\graphicspath{ {./images/} }
\pgfplotsset{compat=1.17}
\begin{document}
\maketitle
\tableofcontents
\newpage
\part*{Lecture 1}
\section{Uniform convergence and uniform continuity}
\subsection{Pointwise and uniform convergence}
Recall that in $\mathbb{R}$ or $\mathbb{C}$ we write $x_{n} \rightarrow x$ as $n \rightarrow \infty$ if
\[
\forall \varepsilon>0 \quad \exists N \in \mathbb{N} \quad \forall n \geqslant N \quad\left|x_{n}-x\right|<\varepsilon
\]
The aim is to define $f_{n} \rightarrow f$ for functions $f_{n}$ and $f$.

\begin{definition}
    We have a set $S$ and functions $f_{n}: S \rightarrow \mathbb{R}, n \in \mathbb{N}$, and $f: S \rightarrow \mathbb{R}$. Then $f_{n} \rightarrow f$ pointwise on $S$ as $n \rightarrow \infty$ if for every $x \in S$, the real sequence $\left(f_{n}(x)\right)_{n}$ converges to $f(x)$. In symbols:
    \[
    \forall x \in S \quad \forall \varepsilon>0 \quad \exists N \in \mathbb{N} \quad \forall n \geqslant N \quad\left|f_{n}(x)-f(x)\right|<\varepsilon
    \]
\end{definition}

\begin{remark}\
    \begin{enumerate}
        \item  $N$ can depend on $\varepsilon$ and $x$.
        \item Can replace $\mathbb{R}$ with $\mathbb{C}$.
        \item It is possible that $f_{n} \rightarrow f$ pointwise on some subset $T \subset S$. In the definition replace $\forall x \in S$ with $\forall x \in T$.
        \item The pointwise limit of a sequence of functions $(f_n)$ is unique if it exists.
    \end{enumerate}
\end{remark}

\begin{example}\
    \begin{enumerate}
        \item $f_{n}(x)=x^{n}$ for $x \in[0,1], n \in \mathbb{N}$.
        
        For $0 \leqslant x<1$, we have $f_{n}(x)=x^{n} \rightarrow 0$ as $n \rightarrow \infty$.
        Also, $f_{n}(1)=1 \rightarrow 1$ as $n \rightarrow \infty$. So $f_{n} \rightarrow f$ pointwise on $[0,1]$, where
        \[
        f(x)= \begin{cases}0 & \text { if } 0 \leqslant x<1 \\ 1 & \text { if } x=1\end{cases}
        \]
        \item $f_{n}(x)=x^{2} \cdot e^{-n x}$ for $x \in[0, \infty)$, and $n \in \mathbb{N}$. For $x>0$
        \[
        0 \leqslant f_{n}(x)=\frac{x^{2}}{e^{n x}}=\frac{x^{2}}{1+n x+\frac{n^{2} x^{2}}{2}+\ldots} \leqslant \frac{x^{2}}{n x}=\frac{x}{n}
        \]
        Hence $f_{n}(x) \rightarrow 0$ as $n \rightarrow \infty$. Same is true $x=0$. Thus, $f_{n} \rightarrow 0$ pointwise on $[0, \infty)$.
    \end{enumerate}
\end{example}

\begin{definition}
    We are given a set $S$ and functions $f_{n}: S \rightarrow \mathbb{R}, n \in \mathbb{N}$, and $f: S \rightarrow \mathbb{R}$. Then $f_{n} \rightarrow f$ uniformly on $S$ as $n \rightarrow \infty$ if
    \[
    \forall \varepsilon>0 \quad \exists N \in \mathbb{N} \quad \forall n \geqslant N \quad \forall x \in S \quad\left|f_{n}(x)-f(x)\right|<\varepsilon
    \]
    \begin{center}
        \includegraphics[scale=0.11]{at1.jpeg}
    \end{center}
\end{definition}

\begin{remark}\
    \begin{enumerate}
        \item $ N $ only depends on $\epsilon$!
        \item Uniform convergence implies pointwise convergence.
        \item Can replace $\mathbb{R}$ with $\mathbb{C}$ and can restrict to a subset of the domain.
        \item An equivalent definition of $f_{n} \rightarrow f$ uniformly on $S$ :
        \[
        \forall \varepsilon>0 \quad \exists N \in \mathbb{N} \quad \forall n \geqslant N \quad \sup _{x \in S}\left|f_{n}(x)-f(x)\right|<\varepsilon
        \]
        Even shorter: $\sup _{x \in S}\left|f_{n}(x)-f(x)\right| \rightarrow 0$ as $n \rightarrow \infty$
        \item The uniform limit $f$ if exists is unique.
    \end{enumerate}
\end{remark}

\begin{example}
\begin{enumerate}
    \item $f_{n}(x)=x^{2} \cdot \mathrm{e}^{-n x}$ for $x \in[0, \infty)$, and $n \in \mathbb{N}$. We saw that $f_{n} \rightarrow f=0$ pointwise on $[0, \infty)$.
    \[
    0 \leqslant f_{n}(x)=\frac{x^{2}}{e^{n x}}=\frac{x^{2}}{1+n x+\frac{n^{2} x^{2}}{2}+\ldots} \leqslant \frac{x^{2}}{n x}=\frac{x}{n}
    \]
    Thus,
    \[
    \sup _{x \in[0, \infty)}\left|f_{n}(x)-f(x)\right|=\sup _{x \in[0, \infty)} f_{n}(x) \leqslant \frac{2}{n^{2}} \rightarrow 0 \quad \text { as } \quad n \rightarrow \infty
    \]
    So $f_{n} \rightarrow 0$ uniformly on $[0, \infty)$.

    Could have used differentiation above to find the supremum, but the above method of finding an upper bound is better.
    \item $f_{n}(x)=x^{n}$ for $x \in[0,1], n \in \mathbb{N}$.
    We know that $f_{n} \rightarrow f$ pointwise on $[0,1]$, where
    \[
        f(x)= \begin{cases}0 & \text { if } 0 \leqslant x<1 \\ 1 & \text { if } x=1\end{cases}.
    \]
    Let $\varepsilon=1 / 2$. For any $n \in \mathbb{N}$, setting $x=\varepsilon^{1 / n}$, we have $f_{n}(x)=\varepsilon$, and so $\left|f_{n}(x)-f(x)\right| \geqslant \varepsilon$. So $f_{n} \not \rightarrow  f$ uniformly on $[0,1]$.

    Better: Since $f_{n}(1)=1$ and $f_{n}$ is continuous, there exist $\delta>0$ such that $\left|f_{n}(x)-1\right|<1 / 2$ for $x \in(1-\delta, 1+\delta)$. So for any $x \in[0,1]$ with $1-\delta<x<1$, we have $\left|f_{n}(x)-f(x)\right| \geqslant \varepsilon$.
\end{enumerate}
\end{example}

Q: Does $\left(f_{n}\right)$ converge uniformly on $S ?$
\subsubsection*{Strategy}
First check if $\left(f_{n}\right)$ converges pointwise on $S$.

If it doesn't, then $\left(f_{n}\right)$ does not converge uniformly on $S$.

If it does, then compute the pointwise limit $f$, and then it remains to check whether
\[
\sup _{x \in S}\left|f_{n}(x)-f(x)\right| \rightarrow 0 \quad \text { as } \quad n \rightarrow \infty
\]

If the above holds, then $f_{n} \rightarrow f$ uniformly on $S$, otherwise $\left(f_{n}\right)$ does not converge uniformly on $S$.

\begin{remark}
    What does it mean that $f_{n} \not \rightarrow f$ uniformly on $S$? We have to negate the sentence
    \[
    \forall \varepsilon>0 \quad \exists N \in \mathbb{N} \quad \forall n \geqslant N \quad \forall x \in S \quad\left|f_{n}(x)-f(x)\right|<\varepsilon
    \]
    The negation is
    \[
    \exists \varepsilon>0 \quad \forall N \in \mathbb{N} \quad \exists n \geqslant N \quad \exists x \in S \quad\left|f_{n}(x)-f(x)\right| \geqslant \varepsilon
    \]
\end{remark}
\newpage
\part*{Lecture 2}
\subsection{Continuity and boundedness}
The next theorem says that uniform limit of continuous functions is continuous.
\begin{theorem}\label{thm:1}
    Let $S$ be a subset of $\mathbb{R}$ or $\mathbb{C}$. Assume $f_{n} \rightarrow f$ uniformly on $S$. If $f_{n}$ is continuous for every $n \in \mathbb{N}$, then $f$ is continuous.
\end{theorem}

\textsf{\textbf{Idea}}: Given $a \in S$, we want that $f(x) \approx f(a)$ provided $x \approx a$. We first choose large $n$ so that $f_{n}(x) \approx f(x)$ for every $x \in S$. Since $f_{n}$ is continuous, we have $f_{n}(x) \approx f_{n}(a)$ provided $x \approx a$. Thus, if $x \approx a$, then $f(x) \approx f_{n}(x) \approx f_{n}(a) \approx f(a)$.

\begin{proof}
    Fix $ a\in S, \epsilon>0 $. The goal is to find a $ \delta $ s.t. 
    \[
        \forall x\in S\quad |x-a|<\delta \Longrightarrow |f(x)-f(a)|<\epsilon.
    \]
    Since $ f_n\to f $ uniformly on $S$, $ \exists n\in \mathbb{N} $ such that 
    \[
        \forall x\in S\quad |f_n(x)-f(x)|<\epsilon.
    \]
    Since $f_n$ is continuous, $ \exists \delta>0 $, 
    \[
        \forall x\in S\quad |x-a|<\delta \Longrightarrow |f_n(x)-f_n(a)|<\epsilon.
    \]
    Thus for any $x\in S$, if $ |x-a|<\delta $,
    \[
        |f(x)-f(a)|\le |f(x)-f_n(x)|+|f_n(x)-f_n(a)|+|f_n(a)-f(a)|<3\epsilon.\qedhere
    \]
\end{proof}
\begin{remark}
    \begin{enumerate}
        \item It is not true for pointwise convergence. A counterexample is $x^n$.
        \item The result does not extend to differentiability.
        \item It follows from Theorem 1.1 that $x^{n}$ does not converge uniformly on $[0,1]$.
        \item The above proof is sometimes called a $3 \varepsilon$-proof.
        \item $\displaystyle \lim _{x \rightarrow a} \lim _{n \rightarrow \infty} f_{n}(x)=\lim _{x \rightarrow a} f(x)=f(a)=\lim _{n \rightarrow \infty} f_{n}(a)=\lim _{n \rightarrow \infty} \lim _{x \rightarrow a} f_{n}(x)$. i.e. we can \textit{swap limits} for uniformly convergent functions.
    \end{enumerate}
\end{remark}
\begin{lemma}\label{lemma 2}
    Let $S$ be any set and $f_{n}$ be a bounded function on $S$ for every $n \in \mathbb{N}$. If $f_{n} \rightarrow f$ uniformly on $S$, then $f$ is also bounded on $S$.
\end{lemma}
\begin{proof}
    Fix $n \in \mathbb{N}$ so that $\left|f_{n}(x)-f(x)\right| \leqslant 1$ for every $x \in S$. We can do this, since $f_{n} \rightarrow f$ uniformly on $S$. Since $f_{n}$ is bounded, there is an $M \in \mathbb{R}$ such that $\left|f_{n}(x)\right| \leqslant M$ for every $x \in S$. It follows that for every $x \in S$, we have
    \[
    |f(x)| \leqslant\left|f(x)-f_{n}(x)\right|+\left|f_{n}(x)\right| \leqslant 1+M
    \]
    Thus, $f$ is bounded by $M+1$.
\end{proof}
\subsection{Integrability and differentiability}

Before the next theorem, we recall some definitions and results from Analysis I. Assume $f:[a, b] \rightarrow \mathbb{R}$ is a bounded function. Given a dissection $\mathcal{D}: a=x_{0}<x_{1}<\cdots<x_{n}=b$ of $[a, b]$, the upper and lower sums of $f$ with respect to $\mathcal{D}$ are defined as
\[
U_{\mathcal{D}}(f)=\sum_{k=1}^{n}\left(x_{k}-x_{k-1}\right) \sup _{\left[x_{k-1}, x_{k}\right]} f \quad \text { and } \quad L_{\mathcal{D}}(f)=\sum_{k=1}^{n}\left(x_{k}-x_{k-1}\right) \inf _{\left[x_{k-1}, x_{k}\right]} f
\]
Riemann's criterion states that $f$ is integrable if and only if for every $\varepsilon>0$ there is a dissection $\mathcal{D}$ of $[a, b]$ such that
\[
U_{\mathcal{D}}(f)-L_{\mathcal{D}}(f)=\sum_{k=1}^{n}\left(x_{k}-x_{k-1}\right)\left(\sup _{\left[x_{k-1}, x_{k}\right]} f-\inf _{\left[x_{k-1}, x_{k}\right]} f\right)<\varepsilon
\]
An easy exercise shows that for an interval $I \subset[a, b]$ we have
\[
\sup _{I} f-\inf _{I} f=\sup _{x, y \in I}(f(x)-f(y))=\sup _{x, y \in l}|f(x)-f(y)|
\]
(This quantity is sometimes called the oscillation of $f$ on $I$.)

\begin{theorem}\label{theorem 3}
    Assume $f_{n}:[a, b] \rightarrow \mathbb{R}$ is Riemann-integrable for every $n \in \mathbb{N}$. If $f_{n} \rightarrow f$ uniformly on $[a, b]$, then $f$ is also Riemann-integrable on $[a, b]$, and moreover
    \[
    \int_{a}^{b} f_{n} \rightarrow \int_{a}^{b} f \quad \text { as } n \rightarrow \infty.
    \]
\end{theorem}
\begin{proof}
    We are going to prove that $f$ is bounded and that it satisfies Riemann's criterion. By definition of integrability, each $f_{n}$ is bounded, and hence so is $f$ by Lemma \ref{lemma 2}.
    Next, fix $\varepsilon>0$. Since $f_{n} \rightarrow f$ uniformly on $[a, b]$, we can fix $n \in \mathbb{N}$ so that $\left|f_{n}(x)-f(x)\right|<\varepsilon$ for all $x \in[a, b]$. Since $f_{n}$ is integrable, it satisfies Riemann's criterion, so there is a dissection $\mathcal{D}$ of $[a, b]$ such $U_{\mathcal{D}}\left(f_{n}\right)-L_{\mathcal{D}}\left(f_{n}\right)<\varepsilon$. If $I$ is one of the sub-intervals of $\mathcal{D}$, then for any $x, y \in I$ we have
    \[
    \begin{aligned}
    |f(x)-f(y)| & \leqslant\left|f(x)-f_{n}(x)\right|+\left|f_{n}(x)-f_{n}(y)\right|+\left|f_{n}(y)-f(y)\right| \\
    &<\left|f_{n}(x)-f_{n}(y)\right|+2 \varepsilon
    \end{aligned}
    \]
    It follows that
    \[
    \sup _{x, y \in I}|f(x)-f(y)| \leqslant \sup _{x, y \in I}\left|f_{n}(x)-f_{n}(y)\right|+2 \varepsilon
    \]
    Multiplying both sides with the length of $I$ and summing over all sub-intervals $I$ of $\mathcal{D}$, we obtain
    \[
    U_{\mathcal{D}}(f)-L_{\mathcal{D}}(f) \leqslant U_{\mathcal{D}}\left(f_{n}\right)-L_{\mathcal{D}}\left(f_{n}\right)+2 \varepsilon(b-a)<(2(b-a)+1) \varepsilon.
    \]
    So $f$ satisfies Riemann's criterion, and thus, $f$ is integrable.
    Finally, we estimate
    \[
    \left|\int_{a}^{b} f_{n}-\int_{a}^{b} f\right| \leqslant \int_{a}^{b}\left|f_{n}-f\right| \leqslant(b-a) \sup _{[a, b]}\left|f_{n}-f\right| \rightarrow 0 \quad \text { as } n \rightarrow \infty .\qedhere
    \]
\end{proof}
\begin{remark}
    The conclusion of Theorem \ref{theorem 3} says: $\int_{a}^{b} \lim _{n \rightarrow \infty} f_{n}(x) \mathrm{d} x=\lim _{n \rightarrow \infty} \int_{a}^{b} f_{n}(x) \mathrm{d} x$. i.e. for uniformly convergent functions we can swap limits and integrals.
\end{remark}
\begin{corollary}\label{col:4}
    Let $f_{n}:[a, b] \rightarrow \mathbb{R}$ be an integrable function for each $n \in \mathbb{N}$. If $\sum_{n=1}^{\infty} f_{n}(x)$ converges uniformly on $[a, b]$, then $\sum_{n=1}^{\infty} f_{n}(x)$ defines an integrable function on $[a, b]$, and moreover
    \[
    \int_{a}^{b} \sum_{n=1}^{\infty} f_{n}(x) \mathrm{d} x=\sum_{n=1}^{\infty} \int_{a}^{b} f_{n}(x) \mathrm{d} x.
    \]
\end{corollary}
\begin{proof}
    Define $F_{n}(x)=\sum_{k=1}^{n} f_{k}(x)$ for $x \in[a, b]$ and $n \in \mathbb{N}$. To say that $\sum_{n=1}^{\infty} f_{n}(x)$ converges uniformly on $[a, b]$ means that $\left(F_{n}\right)$ converges uniformly on $[a, b]$. So for each $x \in[a, b]$, the series $\sum_{n=1}^{\infty} f_{n}(x)$ is convergent, and the function $x \mapsto \sum_{n=1}^{\infty} f_{n}(x)$ is the uniform limit of $\left(F_{n}\right)$ on $[a, b]$.

    We know that each $F_{n}$ is integrable and $\int_{a}^{b} F_{n}=\sum_{k=1}^{n} \int_{a}^{b} f_{k}$. So by Theorem 3 , the function $\sum_{n=1}^{\infty} f_{n}(x)$ is integrable and
    \[
    \int_{a}^{b} \sum_{n=1}^{\infty} f_{n}(x) \mathrm{d} x=\lim _{n \rightarrow \infty} \int_{a}^{b} F_{n}(x) \mathrm{d} x=\lim _{n \rightarrow \infty} \sum_{k=1}^{n} \int_{a}^{b} f_{k}(x) \mathrm{d} x=\sum_{k=1}^{\infty} \int_{a}^{b} f_{k}(x) \mathrm{d} x.\qedhere
    \]
\end{proof}
\begin{theorem}\label{thm:5}
    Let $\left(f_{n}\right)$ be a sequence of continuously differentiable functions on $[a, b]$. Assume further that
    \begin{enumerate}
        \item $\sum_{n=1}^{\infty} f_{n}^{\prime}(x)$ converges uniformly on $[a, b]$;
        \item there exists $c \in[a, b]$ such that $\sum_{n=1}^{\infty} f_{n}(c)$ converges.
    \end{enumerate}
    Then $\sum_{n=1}^{\infty} f_{n}(x)$ converges uniformly to a continuously differentiable function $f$ on $[a, b]$, and moreover, we have
    \[
    f^{\prime}(x)=\sum_{n=1}^{\infty} f_{n}^{\prime}(x) \quad \text { for all } x \in[a, b].
    \]
\end{theorem}
\begin{note}
    Informally, $\displaystyle \frac{\mathrm{d}}{\mathrm{d}x}\left( \sum_{n=1}^{\infty}f_n(x) \right) = \sum_{n=1}^{\infty}\frac{\mathrm{d}f_n}{\mathrm{d}x}  $. 
\end{note}
\begin{proof}
    Let $g(x)=\sum_{n=1}^{\infty} f_{n}^{\prime}(x)$ for $x \in[a, b]$. Idea: solve the equation $f^{\prime}=g$ with initial condition $f(c)=\sum_{n=1}^{\infty} f_{n}(c)$.

    Since $\sum_{n=1}^{\infty} f_{n}^{\prime}(x)$ converges uniformly to $g(x)$, and since $f_{n}^{\prime}$ is continuous for every $n \in \mathbb{N}$, by Theorem \ref{thm:1}, $ g$ is continuous, and hence integrable. Let $\lambda=\sum_{n=1}^{\infty} f_{n}(c)$ and define
    \[
    f(x)=\lambda+\int_{c}^{x} g(t)\, \mathrm{d} t \quad \text { for } x \in[a, b]
    \]
    Since $g$ is continuous, by the Fundament Theorem of Calculus (FTC) $f$ is differentiable with $f^{\prime}=g$, and moreover $f(c)=\lambda$. By the FTC we also have
    \[
    f_{k}(x)=f_{k}(c)+\int_{c}^{x} f_{k}^{\prime}(t)\, \mathrm{d} t \quad \text { for } x \in[a, b], k \in \mathbb{N}
    \]
    Given $\varepsilon>0$, by our assumptions there exists $N \in \mathbb{N}$ such that
    \[
    \begin{array}{rll}
    \displaystyle \left|\lambda-\sum_{k=1}^{n} f_{k}(c)\right| & <\varepsilon & \forall n \geqslant N \\[15pt]
    \displaystyle \left|g(t)-\sum_{k=1}^{n} f_{k}^{\prime}(t)\right| & <\varepsilon & \forall n \geqslant N \quad \forall t \in[a, b]
    \end{array}
    \]
    It follows that for all $n \geqslant N$ and for all $x \in[a, b]$ we have
    \[
    \begin{aligned}
    \left|f(x)-\sum_{k=1}^{n} f_{k}(x)\right| &=\left|\lambda+\int_{c}^{x} g(t)\, \mathrm{d} t-\sum_{k=1}^{n}\left(f_{k}(c)+\int_{c}^{x} f_{k}^{\prime}(t) \,\mathrm{d} t\right)\right| \\
    &=\left|\lambda-\sum_{k=1}^{n} f_{k}(c)+\int_{c}^{x}\left(g(t)-\sum_{k=1}^{n} f_{k}^{\prime}(t)\right) \mathrm{d} t\right| \\
    & \leqslant\left|\lambda-\sum_{k=1}^{n} f_{k}(c)\right|+\left|\int_{c}^{x}\left(g(t)-\sum_{k=1}^{n} f_{k}^{\prime}(t)\right) \mathrm{d} t\right| \\
    &<\varepsilon+(b-a) \varepsilon.
    \end{aligned}
    \]
    This shows that $\sum_{k=1}^{n} f_{k}(x) \rightarrow f(x)$ uniformly on $[a, b]$. We have already seen that $f$ is differentiable and $f^{\prime}=g$ is continuous.
\end{proof}
\newpage
\part*{Lecture 3}
\subsection{Uniform Cauchy}
We recall from Analysis I: a scalar sequence $\left(x_{n}\right)$ is Cauchy if
\[
\forall \varepsilon>0 \quad \exists N \in \mathbb{N} \quad \forall m, n \geqslant N \quad\left|x_{m}-x_{n}\right|<\varepsilon.
\]
The General Principle of Convergence (GPC): every Cauchy sequence is convergent.
\begin{definition}
    Let $\left(f_{n}\right)$ be a sequence of scalar functions on a set $S$. We say $\left(f_{n}\right)$ is uniformly Cauchy on $S$ if
    \[
    \forall \varepsilon>0 \quad \exists N \in \mathbb{N} \quad \forall m, n \geqslant N \quad \forall x \in S \quad\left|f_{m}(x)-f_{n}(x)\right|<\varepsilon.
    \]
\end{definition}

\begin{theorem}[General Principle of Uniform Convergence (GPUC)]\label{thm:6}
    If $\left(f_{n}\right)$ is a uniformly Cauchy sequence of scalar functions on a set $S$, then $\left(f_{n}\right)$ converges uniformly to some function $f$ on $S$.
\end{theorem}
\end{document}